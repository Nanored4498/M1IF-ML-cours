\documentclass[11pt,oneside]{book}

\usepackage[french]{babel}
\usepackage[utf8]{inputenc}  
\usepackage[T1]{fontenc}
\usepackage{listings}
\usepackage{comment}
\usepackage[left=2.5cm,right=2.5cm,top=3cm,bottom=2.65cm]{geometry}
\usepackage{graphicx}
\usepackage{tabularx}
\usepackage{amsmath,amssymb,amsfonts}
\usepackage{bbm, bm}
\usepackage{stmaryrd}
\usepackage{mathtools}
\usepackage{hyperref}
\usepackage{tikz}
\usepackage{tabularx}
\usepackage{makecell}
\usepackage{color}
\usepackage{fancybox}
\usepackage[thmmarks,amsmath]{ntheorem}
\usepackage{minitoc}
\usepackage{titletoc}
\usepackage{tikz}
\usepackage{algpseudocode}
\usepackage[]{algorithm2e}
\usepackage{wrapfig}
\usepackage{kpfonts}

\graphicspath{{images/}}

\DeclareMathOperator*{\argmax}{arg\,max}
\DeclareMathOperator*{\argmin}{arg\,min}

\usetikzlibrary{arrows.meta}
\usetikzlibrary{arrows}
\usetikzlibrary{decorations.pathreplacing}
\mathtoolsset{showonlyrefs=true}

\addto\extrasfrench{%
	\def\subsectionautorefname{\S}%
	\def\sectionautorefname{\S}%
}

\definecolor{darkWhite}{rgb}{0.94,0.94,0.94}
\definecolor{blue}{rgb}{0.12,0.16,0.53}
\definecolor{green}{rgb}{0.25,0.28,0.06}
\definecolor{purple}{rgb}{0.5, 0.1, 0.5}
\definecolor{yellow}{rgb}{0.8, 0.7, 0.1}
\definecolor{greenTikz}{rgb}{0.16,0.53,0.12}
\definecolor{red}{rgb}{0.71,0.19,0.11}
\definecolor{redLight}{rgb}{0.85,0.24,0.15}
\definecolor{darkPurple}{rgb}{0.2,0.05,0.18}
\definecolor{whiteGray}{rgb}{0.92,0.96,0.95}

\newcounter{sss}[subsection]
\renewcommand\thesection{\textcolor{red}{\Roman{section} -}}
\renewcommand\thesubsection{\textcolor{blue}{\arabic{subsection}/}}
\renewcommand\thesss{\textcolor{green}{\alph{sss}.}}
\renewcommand\thechapter{\Alph{chapter}}
\newcommand{\sect}[1]{\section{\textcolor{red}{#1}}}
\newcommand{\subs}[1]{\subsection{\textcolor{blue}{#1}}}
\newcommand{\subsubs}[1]{
	\stepcounter{sss}
	\subsubsection{\textcolor{green}{\thesss~#1}}
}

\newcommand{\mybox}[4]{
	\begin{center}
		\boxput*(0,1){\colorbox{darkWhite}{\textbf{#1}}}{
			\setlength{\fboxsep}{12pt}
			\fcolorbox{#2}{#3}{
				\begin{Bflushleft}
					\begin{minipage}{0.908\linewidth}
						\vspace{2mm} \par \textcolor{#2}{#4}
					\end{minipage}
				\end{Bflushleft}
			}
		}
	\end{center}
}

\newcommand{\mytitle}[1]{
	\title{
		\fcolorbox{black}{whiteGray}{
			\begin{Bflushleft}
				\\ \huge \textbf{\textcolor{red}{#1}}
			\end{Bflushleft}
		}
	}
}

\setcounter{tocdepth}{1}
\setcounter{minitocdepth}{3}
\nomtcrule
\titlecontents*{chapter}[0pt]{}
{\bfseries\chaptername\ \thecontentslabel\quad}{}
{\bfseries\hfill\contentspage}
\newcommand{\mytoc}{
	\renewcommand{\contentsname}{}
	\mybox{Table des matières}{darkPurple}{darkWhite}{
		\vspace{-40mm}\tableofcontents}
}
\newcommand{\myminitoc}{
	\mybox{Table des matières}{darkPurple}{darkWhite}{
		\vspace{-9mm}\minitoc \vspace{-8mm}}
}


\newcommand{\DEF}[1]{\vspace{1mm} \mybox{Définition}{blue}{white}{#1}}
\newcommand{\REM}[1]{\vspace{1mm} \mybox{Remarque}{darkPurple}{white}{#1}}

\newcounter{propNum}
\newcommand{\PROP}[2][]{
	\stepcounter{propNum}
	\vspace{1mm}
	\mybox{Proposition \thepropNum #1}{red}{white}{#2}
}
\newcommand{\dem}{\paragraph{Démonstration}}
\newcommand{\findem}{\hfill $\blacksquare$}
\newcommand{\exe}{\paragraph{\textit{\textcolor{green}{Exemple}}}}

\newcommand{\N}{\mathbb{N}}
\newcommand{\trans}{\mathsf{T}}
\newcommand{\R}{\mathbb{R}}
\newcommand{\Pp}{\mathbb{P}}
\newcommand{\E}{\mathbb{E}}
\newcommand{\X}{\mathcal{X}}
\newcommand{\Y}{\mathcal{Y}}
\newcommand{\Z}{\mathcal{Z}}
\newcommand{\dist}{\mathcal{D_Z}}
\newcommand{\Hyp}{\mathcal{H}}
\newcommand{\trisk}{\mathcal{R}^l}
\newcommand{\erisk}{\hat{\mathcal{R}}^l}
\newcommand{\vrisk}{\tilde{\mathcal{R}}}

\mytitle{Machine Learning}
\author{
	Notes écrites par \\
	Yoann Coudert--Osmont \\
	\texttt{yoann.coudert-osmont@ens-lyon.fr}
	\and
	D'après un cours de \\
	Marc Sebban \\
	University of Jean Monnet Saint-Étienne
}
\date\today

\begin{document}
	
	\dominitoc[n]
	\maketitle
	\mytoc
	
	\chapter{Introduction, apprentissage supervisé, bornes}

\myminitoc

\sect{Introduction}

\paragraph{Qu'est ce que le machine learning ?}
Le machine learning est le développement d'algorithmes qui apprennent tout seul à partir de données. On distingue deux catégorie :
\begin{itemize}
	\item L'apprentissage supervisé : qui apprend avec des données étiquetées afin de faire de la classification, de la régression ou encore de la hiérarchisation.
	\item  L'apprentissage non supervisé : qui trouve la structure d'un jeu de données afin de faire du clustering ou de la réduction de dimensions.
\end{itemize}
Les applications principales du machine learning sont alors la vision par ordinateur, la robotique, la reconnaissance vocale, le traitement du langage, etc ...

\sect{Apprentissage supervisé}

\DEF{
	Dans la suite on utilisera les notations suivantes :
	\begin{itemize}
		\item On pose $\bm{S = \{ z_i = (x_i, y_i) \}_{i = 1}^m}$ un ensemble de $\bm{m}$ exemples d'entraînement indépendants et identiquement distribués selon une une distribution inconnue $\bm{\dist}$ sur l'espace $\bm{\Z = \X \times \Y}$.
		\item Les valeurs $x_i \in \bm{\X}$ sont généralement des vecteurs de $\R^d$ dont les composantes sont appelées les \textbf{features}.
		\item Les valeurs $y_i \in \bm{\Y}$ se trouvent dans l'ensemble discret des \textbf{classes/étiquettes} (typiquement $\Y = \{ -1, +1 \}$ en classification binaire) ou dans un ensemble continue dans le cas de régressions.
		\item Finalement on cherche une \textbf{fonction cible} $\bm{f}$ tel que $\bm{\forall (x, y) \in \dist, \, y = f(x)}$.
	\end{itemize}
}

\DEF{
	Un \textbf{algorithme d'apprentissage supervisé} $\bm{L}$ prend en entrée $S$ et retourne un modèle ou une classification $\bm{h \in \Hyp}$ le plus proche possible de $f$.
}

\exe
Si on prend pour $f$ la fonction qui retourne {\color{red}$y = +1$} si $x_1^2 + x_2^2 < R^2$ et {\color{blue}$y = -1$} sinon, alors voici le résultat que l'on peut obtenir :
\begin{center}
	\begin{tikzpicture}[thick, scale=1.2]
		\draw[greenTikz, opacity=0.4] (0, 0) circle (1);
		\node[greenTikz] at (0.8, -0.9) {$f$};
		\draw[blue] (-2, 0) -- (2, 0) node[below left, black] {$x_1$};
		\draw[blue] (0, -2) -- (0, 1.8) node[above, black] {$x_2$};
		\draw[fill, red] (0.2, 0.7) circle (0.08);
		\draw[fill, red] (-0.4, -0.5) circle (0.08);
		\draw[fill, red] (0.2, -0.3) circle (0.08);
		\draw[fill, red] (-0.3, 0.3) circle (0.08);
		\draw[fill, red] (0.4, 0.2) circle (0.08);
		\draw[orange, very thick] (0.2, -1.5)
			.. controls (-0.5, -1) and (-1, -0.2) .. (-0.95, -0.05)
			.. controls (-1, 0.2) and (-0.4, 1) .. (0.3, 0.9)
			.. controls (0.9, 0.75) and (1, 0) .. (1.1, -0.4)
			.. controls (1.15, -0.6) and (1.05, -1.2) .. (0.8, -1.5) node[below] {$h$};
		\draw[fill, blue] (-1.1, 0.3) circle (0.08);
		\draw[fill, blue] (-1.2, -0.1) circle (0.08);
		\draw[fill, blue] (-1.3, -0.5) circle (0.08);
		\draw[fill, blue] (-1.2, -0.8) circle (0.08);
		\draw[fill, blue] (-0.8, -1.2) circle (0.08);
		\draw[fill, blue] (-0.3, -1.5) circle (0.08);
		\draw[fill, blue] (-0.6, 1) circle (0.08);
		\draw[fill, blue] (-0.9, 0.8) circle (0.08);
		\draw[fill, blue] (-0.95, 1.2) circle (0.08);
		\draw[fill, blue] (-0.2, 1.3) circle (0.08);
		\draw[fill, blue] (0.3, 1.1) circle (0.08);
		\draw[fill, blue] (0.7, 0.9) circle (0.08);
		\draw[fill, blue] (1.1, 0.5) circle (0.08);
		\draw[fill, blue] (1.3, 0.1) circle (0.08);
		\draw[fill, blue] (1.3, -0.4) circle (0.08);
		\draw[fill, blue] (1.25, -1.2) circle (0.08);
	\end{tikzpicture}
\end{center}
Ici $h$ convient aux données d'entraînement mais n'est toujours pas bon pour la généralisation.

\paragraph{Conjecture}
Plus l'ensemble $S$ sera grand, plus la fonction $h$ sera proche de $f$.

\paragraph{Malédiction de la dimensionnalité}
Quand le nombre de features augmente, le nombre $m$ d'exemples d'entraînement nécessaires pour généralisé de manière assez précise augmente exponentiellement.

\DEF{
	En statistiques, l'\textbf{overfitting} est le phénomène où le modèle obtenu est trop complexe. Il peu avoir trop de degrés de liberté par exemple. En revanche l'\textbf{underfitting} est lorsque le modèle n'arrive pas à trouver la tendance des données. 
}

\subs{Risque et fonction de perte}

\DEF{
	En théorie, on aime considérer la meilleure hypothèse $h^* \in \Hyp$. En se donnant une \textbf{fonction de perte} $l : \Hyp \times \Z \rightarrow \R$ mesurant le degré d'accord entre $h(x)$ et $y$, le \textbf{vrai risque} ou \textbf{erreur de généralisation} $\trisk(h)$ est défini ainsi :
	$$ \trisk(h) = \E_{z \sim \dist} l(h, z) = \int_z f_\dist(z) l(h, z)$$
	$$ h^* = \argmin_{h \in \Hyp} \trisk(h) $$
	\vspace{-5mm}
}

Malheureusement, $\trisk(h)$ ne peut pas être calculé car $\dist$ est inconnu. On peut seulement calculé le \textbf{risque empirique} sur $S$. C'est à dire :
$$ \erisk(h) = \E_{z \sim S} l(h, z) = \frac{1}{m} \sum_{i = 1}^{m} l(h, z_i) $$
Ainsi le but de l'algorithme d'apprentissage supervisé est de trouver le modèle $\displaystyle h = \argmin_{h_i \in \Hyp} \erisk(h_i)$.

\exe
La fonction de perte la plus naturelle pour la classification binaire est le 0/1 loss.
$$ l_{0/1}(h, z) = \left\{ \begin{array}{ll}
	1 & \text{si } yh(x) < 0 \\
	0 & \text{sinon}
\end{array}
\right. $$
Ainsi $\mathcal{R}^{l_{0/1}}(h)$ est la proportion de mauvaises prédictions. \\
Malheureusement, à cause de la non-convexité et de la non-différentiabilité de cette fonction de perte, minimiser, ou même minimiser approximativement $\mathcal{\hat{R}}^{l_{0/1}}(h)$ est un problème NP-difficile.

\paragraph{Fonctions de perte usuelles} Pour cette raison, on utilise généralement les fonctions de perte convexes suivante :
\begin{itemize}
	\item La \textbf{perte exponentielle} (utilisée en boosting) : $l_{exp}(h, z) = e^{-yh(x)}$
	\item La \textbf{perte logistique} (utilisée en régression logistique) : $l_{log}(h, z) = \ln(1 + e^{-yh(x)})$
	\item La \textbf{perte charnière} (utilisée en SVM) : $l_{hinge}(h, z) = \max(0, 1 - yh(x))$
\end{itemize}
\begin{center}
	\begin{tikzpicture}[yscale=1.25, xscale=1.65, thick]
		\draw (-2, -0.5) node[below] {-2} -- (2, -0.5) node[below] {2};
		\draw (-2, -0.5) node[left] {-0.5} -- (-2, 3.5) node[left] {3.5};
		\draw (0, -0.5) node[] {\tiny |} node[below] {0};
		\draw (-2, 1.5) node[] {\tiny -} node[left] {1.5};
		\draw (-2, 1) node[left] {1} -- (0, 1) -- (0, 0) -- (2, 0);
		\draw[red] (-2, 3) -- (1, 0) -- (2, 0);
		\draw[domain=-1.25:2, smooth, variable=\x, blue] plot ({\x}, {exp(-\x)});
		\draw[domain=-2:2, smooth, variable=\x, greenTikz] plot ({\x}, {ln(1 + exp(-\x))});
		\draw (1.8, 3.3) -- (1.3, 3.3) node[left] {\footnotesize 0/1 loss};
		\draw[red] (1.8, 3) -- (1.3, 3) node[left] {\footnotesize hinge loss};
		\draw[greenTikz] (1.8, 2.7) -- (1.3, 2.7) node[left] {\footnotesize logistic loss};
		\draw[blue] (1.8, 2.4) -- (1.3, 2.4) node[left] {\footnotesize exponential loss};
	\end{tikzpicture}
\end{center}

\subs{Minimisation de risque régularisée}

Trop entraîner l'algorithme sur les données d'entraînement $S$ peut conduire à une mémorisation et à l'overfitting. Le modèle devient compliqué et on risque d'avoir une mauvaise généralisation. Le principe du \textbf{rasoir d'Occam} est "le plus simple est le mieux". Pour appliquer ce principe, on essaye de minimiser les paramètres du modèle. \\
On va donc minimiser le \textbf{risque empirique régularisé} :
$$ \min_{h \in \Hyp} \erisk(h) + \lambda \| h \| $$
On pénalise alors les hypothèses avec une forte norme.

\DEF{
	La norme $l_p$ d'un vecteur $\theta$ d'un espace à $d$ dimensions est défini comme suit :
	\vspace*{-3mm}
	$$ \| \theta \|_p = \left( \sum_{i = 1}^d |\theta_i|^p \right)^{\frac{1}{p}} $$
	\vspace{-5mm}
}

\begin{center}
	\begin{tikzpicture}[thick, >={latex}, scale=0.8]
		\draw[->] (0, 0) -- (2, 0);
		\draw[->] (1, -1) node[below] {$p=0$} -- (1, 1);
		\draw[red, very thick] (0.3, 0) -- (1.7, 0);
		\draw[red, very thick] (1, -0.7) -- (1, 0.7);
		
		\draw[->] (3, 0) -- (5, 0);
		\draw[->] (4, -1) node[below] {$p=0.3$} -- (4, 1);
		\draw[domain=-1:1, smooth, variable=\x, red, very thick]
			plot ({0.8*\x+4}, {0.8 * max(0.01, 1 - abs(\x)^0.3)^(1/0.3)});
			\draw[domain=-1:1, smooth, variable=\x, red, very thick]
			plot ({0.8*\x+4}, {-0.8 * max(0.01, 1 - abs(\x)^0.3)^(1/0.3)});
			
		\draw[->] (6, 0) -- (8, 0);
		\draw[->] (7, -1) node[below] {$p=0.5$} -- (7, 1);
		\draw[domain=-1:1, smooth, variable=\x, red, very thick]
			plot ({0.8*\x+7}, {0.8 * max(0.01, 1 - abs(\x)^0.5)^(1/0.5)});
		\draw[domain=-1:1, smooth, variable=\x, red, very thick]
			plot ({0.8*\x+7}, {-0.8 * max(0.01, 1 - abs(\x)^0.5)^(1/0.5)});
		
		\draw[->] (9, 0) -- (11, 0);
		\draw[->] (10, -1) node[below] {$p=1$} -- (10, 1);
		\draw[red, very thick] (9.2, 0) -- (10, 0.8) -- (10.8, 0) -- (10, -0.8) -- (9.2, 0);
		
		\draw[->] (12, 0) -- (14, 0);
		\draw[->] (13, -1) node[below] {$p=1.5$} -- (13, 1);
		\draw[domain=-1:1, smooth, variable=\x, red, very thick]
			plot ({0.8*\x+13}, {0.8 * max(0.01, 1 - abs(\x)^1.5)^(1/1.5)});
		\draw[domain=-1:1, smooth, variable=\x, red, very thick]
			plot ({0.8*\x+13}, {-0.8 * max(0.01, 1 - abs(\x)^1.5)^(1/1.5)});
		
		\draw[->] (15, 0) -- (17, 0);
		\draw[->] (16, -1) node[below] {$p=2$} -- (16, 1);
		\draw[red, very thick] (16, 0) circle (0.8);
		
		\draw[->] (18, 0) -- (20, 0);
		\draw[->] (19, -1) node[below] {$p=inf$} -- (19, 1);
		\draw[red, very thick] (18.2, 0.8) -- (19.8, 0.8) -- (19.8, -0.8) -- (18.2, -0.8) -- (18.2, 0.8);
	\end{tikzpicture}
\end{center}

La norme $l_2$ est utilisée pour réduire les risques d'overfitting en réduisant les plus grandes valeurs du modèle. La norme $l_1$, elle, permet d'obtenir des modèles creux, c'est à dire avec peu de features.

\exe
Considérons le problème suivant : \vspace{-1mm}
$$ \min_{\theta \in \R^d} \frac{1}{2} \theta^\trans \theta - \theta^\trans x + \lambda \| \theta \|_2^2 $$
Si $\lambda = 0$, alors : \vspace{-1mm}
$$ \dfrac{\partial \frac{1}{2} \theta^\trans \theta - \theta^\trans x}{\partial \theta_j} = 0 \Rightarrow \theta_j - x_j = 0 \Rightarrow \fbox{$\theta_j = x_j$} $$
Si $\lambda \neq 0$, alors : \vspace{-2mm}
$$ \dfrac{\partial \frac{1}{2} \theta^\trans \theta - \theta^\trans x  + \lambda \| \theta \|_2^2}{\partial \theta_j} = 0 \Rightarrow \theta_j - x_j +2\lambda \theta_j = 0 \Rightarrow \fbox{$\theta_j = \dfrac{x_j}{1 + 2\lambda}$} $$
\begin{center}
	\begin{tikzpicture}[>={latex}, thick]
	\draw[->] (-1, 0) -- (5, 0);
	\draw[->] (0, -1) -- (0, 3.5);
	\draw (3, 0) node {\tiny |} node[below] {3};
	\draw (0, 2) node {\tiny -} node[left] {2};
	\draw[fill] (3, 2) circle (0.05) node[below] {$x$};
	\draw[fill=greenTikz, fill opacity=0.5, very thick] (3, 2) circle (1.4^0.5);
	\draw[very thick] (3, 2) circle (1) circle (0.6^0.5) circle (0.2^0.5);
	\node[color=red] at (1, -0.6) {$\lambda = 0$};
	
	\draw[->] (7, 0) -- (13, 0);
	\draw[->] (8, -1) -- (8, 3.5);
	\draw (11, 0) node {\tiny |} node[below] {3};
	\draw (8, 2) node {\tiny -} node[left] {2};
	\draw[fill] (11, 2) circle (0.05) node[below] {$x$};
	\draw[fill=greenTikz, fill opacity=0.5, very thick] (9, 0.67) circle (0.47^0.5);
	\draw[very thick] (9, 0.67) circle (0.33^0.5) circle (0.2^0.5) circle (0.07^0.5);
	\node[color=red] at (9, -0.6) {$\lambda = 1$};
	\draw[red, very thick, dashed] (3, 2) -- (0, 0);
	\draw[red, very thick, dashed] (11, 2) -- (8, 0);
	\draw[red] (9, 0.67) circle (0.08);
	\end{tikzpicture}
\end{center}

\exe
Maintenant on peut prendre la norme $l_1$ pour constater qu'elle engendre bien un vecteur de paramètres creux.
$$ \min_{\theta \in \R^d} \frac{1}{2} \theta^\trans \theta - \theta^\trans x + \lambda \| \theta \|_1 $$
Si $\lambda = 0$, on a vu que $\theta^* = x$. \\
En revanche si $\lambda > 0$, on considère la dérivée partielle à $\theta_j = 0^+$, et à $\theta_j = 0^-$ :
$$ g_+^j = \lambda - x_j \qquad g_-^j = - \lambda - x_j $$
Or $\theta_j^* = 0$ si et seulement si $g_+^j \geqslant 0$ et $g_-^j \leqslant 0$. C'est à dire si $x_j \leqslant \lambda$ et $x_j \geqslant -\lambda$. \\
Donc si $|x_j| \leqslant \lambda$ alors $\theta_j^* = 0$.
\begin{center}
	\begin{tikzpicture}[>={latex}, thick, scale=0.8]
	\draw[fill=greenTikz, fill opacity=0.5, very thick] (3, 2) circle (5^0.5);
	\draw[->] (-1, 0) -- (5, 0);
	\draw[->] (0, -1) -- (0, 3.5);
	\draw (3, 0) node {\tiny |} node[below] {3};
	\draw (0, 2) node {\tiny -} node[left] {2};
	\node[color=red] at (1, -0.6) {$\lambda = 0$};
	\draw[fill] (3, 2) circle (0.05) node[below] {$x$};
	\draw[very thick] (3, 2) circle (3^0.5) circle (1);
	
	\draw[very thick, fill=greenTikz, fill opacity=0.5]
		(11.24, 0) arc (0:116.6:2.24) arc (158:180:5.39) arc (217:222.4:6.71) arc (257.8:299.9:4.58);
	\draw[->] (7, 0) -- (13, 0);
	\draw[->] (8, -1) -- (8, 3.5);
	\draw (11, 0) node {\tiny |} node[below] {3};
	\draw (8, 2) node {\tiny -} node[left] {2};
	\draw[fill] (11, 2) circle (0.05) node[below] {$x$};
	\node[color=red] at (12.4, 3) {$\lambda = 2$};
	\node[color=red] at (12.4, 2.5) {$\theta^*$ est creux};
	\draw[very thick] (10, 0) arc(0:180:1);
	\draw[very thick] (8, 0) arc(-104:-76:4.12);
	\draw[very thick] (10.73, 0) arc(0:125:3^0.5);
	\draw[very thick] (7.8, 0) arc(180:164:27^0.5);
	\draw[very thick] (7.8, 0) arc(217.6:220.5:43^0.5);
	\draw[very thick] (10.73, 0) arc(-66.6:-103:19^0.5);
	\draw[fill, red] (9, 0) circle (0.08) node[above] {$\theta^*$};
	\draw[red, very thick, dashed] (3, 2) -- (1, 0) -- (0, 0);
	\draw[red, very thick, dashed] (11, 2) -- (9, 0) -- (8, 0);
	\end{tikzpicture}
\end{center}

\paragraph{Supprimer des groupes de features}
Voici des normes qui permettent de supprimer les features en groupe :
\begin{center}
	\includegraphics[scale=0.5]{group_sparse.png}
\end{center}
On considère $\{ \mathcal{G}_{k = 1}^K \}$ une partition de $\{ 1, \dots, d \}$. On peur alors définir les normes suivantes :
$$ \| \theta \|_{group} = \sum_{g \in \mathcal{G}} \left( \sum_{j \in g} \theta_j^2 \right)^{\frac{1}{2}} $$
$$ \| \theta \|_{coop} = \sum_{g \in \mathcal{G}} \left[ \left( \sum_{j \in g} [\theta_j]_+^2 \right)^{\frac{1}{2}} + \left( \sum_{j \in g} [\theta_j]_-^2 \right)^{\frac{1}{2}} \right] $$

\newpage
\subs{Contrepartie Biais/Variance}

D'où vient l'erreur de $h \in \Hyp$ ?
\begin{itemize}
	\item Du \textbf{biais inductif}. Rien ne garanti l'égalité entre l'espace cible des concepts $\mathcal{F}$ et la classe d'hypothèse que l'on a choisi $\Hyp$.
	\item De la \textbf{variance}. Comme l'ensemble d'entraînement est fini et choisi aléatoirement selon $\dist$, l'algorithme d'apprentissage ne retourne pas l'hypothèse optimale $h^*$ de $\Hyp$.
\end{itemize}

\begin{center}
	\begin{tikzpicture}[thick, scale=1.3]
		\draw (0, 0.5) -- (4, 0.5) -- (4, 2) -- node[above] {$\Hyp$} (0, 2) -- (0, 0.5);
		\node at (5.5, 0.5) {$\mathcal{F}$};
		\draw[fill] (0.5, 1) circle (0.05) node[above] {$h$}
			-- node[below, sloped] {\footnotesize erreur totale} (3.7, -1)
				circle (0.05) node[right] {$f$}
			-- node[above, sloped] {\footnotesize biais} (2.5, 1)
				circle (0.05) node[above] {$h^*$}
			-- node[above] {\footnotesize Variance} (0.5, 1);
	\end{tikzpicture}
\end{center}
$$ \trisk(h) \leqslant \text{Biais} + \text{Variance} $$
$$ \trisk(h) \leqslant \text{Biais inévitable} + \text{Biais évitable} + \text{Variance} $$
$$ \trisk(h) \leqslant {\color{red}\text{Erreur de Bayes}} + {\color{blue}\text{Biais évitable}} + {\color{blue}\text{Variance}} $$

\DEF{
	L'\textbf{erreur de Bayes} $\epsilon_B$ est le plus petit taux d'erreur pour une hypothèse $h$ :
	$$ \epsilon_B = \sum_i \int_{(x, y) \in R_i \times \bar{C_i}} P(C_i | x) p(x) dx $$
	Où $x$ est une instance avec $y$ pour étiquette et $R_i$ est la région que la fonction de classification $h$ classifie comme $C_i$.
}
\begin{center}
	C'est un $\ll$ sept $\gg$ ou un $\ll$ un $\gg$ ? \\
	\includegraphics[scale=0.4]{one_seven.png}
\end{center}

\newpage
\paragraph{Variance}
$h$ va converger vers $h^*$ si on augmente le nombre d'exemples $m$.
\begin{center}
	\begin{tikzpicture}[thick, scale=0.9]
		\draw[ultra thick, blue, opacity=0.8] (0, -2.5) -- (0, 2.5);
		\draw[ultra thick, blue, opacity=0.8] (-1.6, -2.5) -- (0.25, 2.5);
		\node[below] at (-1.6, -2.5) {$h$};
		\node[below] at (0, -2.5) {$h^*$};
		\draw[<->, red] (-0.3, -2.8) -- node[black, above] {?} (-1.4, -2.8);
		\draw[fill, red] (0.1, -2) circle (0.07);
		\draw[fill, red] (-0.6, -1) circle (0.07);
		\draw[fill, red] (0.3, 0.5) circle (0.07);
		\draw[fill, red] (0.5, 2) circle (0.07);
		\draw[fill, red, opacity=0.3] (-0.5, 0) circle (0.07);
		\draw[fill, red, opacity=0.3] (0.3243, 0.9859) circle (0.07);
		\draw[fill, red, opacity=0.3] (-0.08914, 0.978) circle (0.07);
		\draw[fill, red, opacity=0.3] (2.427, 0.7918) circle (0.07);
		\draw[fill, red, opacity=0.3] (1.4609, 1.212) circle (0.07);
		\draw[fill, red, opacity=0.3] (1.3548, 0.6159) circle (0.07);
		\draw[fill, red, opacity=0.3] (1.403, 0.351) circle (0.07);
		\draw[fill, red, opacity=0.3] (-0.351, 1.57183) circle (0.07);
		\draw[fill, red, opacity=0.3] (0.10869, 1.876) circle (0.07);
		\draw[fill, red, opacity=0.3] (-0.13232, 1.62) circle (0.07);
		\draw[fill, red, opacity=0.3] (0.048154, 0.0303) circle (0.07);
		\draw[fill, red, opacity=0.3] (0.0214, -0.861) circle (0.07);
		\draw[fill, red, opacity=0.3] (-0.10329, -1.902) circle (0.07);
		\draw[fill, red, opacity=0.3] (-0.049, -0.0497) circle (0.07);
		\draw[fill, red, opacity=0.3] (1.3507, -1.7576) circle (0.07);
		\draw[fill, red, opacity=0.3] (-0.4912, -0.3749) circle (0.07);
		\draw[fill, red, opacity=0.3] (1.0138, -0.184) circle (0.07);
		\draw[fill, red, opacity=0.3] (2.0108, 0.5766) circle (0.07);
		\draw[fill, red, opacity=0.3] (2.0224, 1.0361) circle (0.07);
		\draw[fill, red, opacity=0.3] (0.0751, 1.7442) circle (0.07);
		\draw[fill, red, opacity=0.3] (2.0863, 1.7777) circle (0.07);
		\draw[fill, red, opacity=0.3] (-0.0409, -0.1897) circle (0.07);
		\draw[fill, red, opacity=0.3] (0.2889, -0.0546) circle (0.07);
		\draw[fill, red, opacity=0.3] (2.4079, -0.1523) circle (0.07);
		\draw[fill, red, opacity=0.3] (1.8439, 1.2179) circle (0.07);
		
		\draw[fill, blue] (-0.8, 1.5) circle (0.07);
		\draw[fill, blue] (-1.2, 0.9) circle (0.07);
		\draw[fill, blue] (-1.6, -0.8) circle (0.07);
		\draw[fill, blue] (-1.6, -1.7) circle (0.07);
		\draw[fill, blue, opacity=0.3] (-1.9025, 0.0568) circle (0.07);
		\draw[fill, blue, opacity=0.3] (-0.2239, -0.3909) circle (0.07);
		\draw[fill, blue, opacity=0.3] (-0.6662, 0.0906) circle (0.07);
		\draw[fill, blue, opacity=0.3] (0.4667, 0.1391) circle (0.07);
		\draw[fill, blue, opacity=0.3] (-0.0517, -1.9698) circle (0.07);
		\draw[fill, blue, opacity=0.3] (-1.1208, -0.8846) circle (0.07);
		\draw[fill, blue, opacity=0.3] (0.1495, -0.2879) circle (0.07);
		\draw[fill, blue, opacity=0.3] (-2.4247, 0.4716) circle (0.07);
		\draw[fill, blue, opacity=0.3] (-1.9027, -1.2081) circle (0.07);
		\draw[fill, blue, opacity=0.3] (-1.4246, 1.5652) circle (0.07);
		\draw[fill, blue, opacity=0.3] (-1.729, 1.9974) circle (0.07);
		\draw[fill, blue, opacity=0.3] (-0.5392, 1.2422) circle (0.07);
		\draw[fill, blue, opacity=0.3] (0.3168, 0.8305) circle (0.07);
		\draw[fill, blue, opacity=0.3] (-0.2799, 0.1071) circle (0.07);
		\draw[fill, blue, opacity=0.3] (-1.5449, 0.0752) circle (0.07);
		\draw[fill, blue, opacity=0.3] (-1.8949, -0.0877) circle (0.07);
		\draw[fill, blue, opacity=0.3] (-0.9253, -0.423) circle (0.07);
		\draw[fill, blue, opacity=0.3] (0.279, -1.1842) circle (0.07);
		\draw[fill, blue, opacity=0.3] (-0.152, 0.5651) circle (0.07);
		\draw[fill, blue, opacity=0.3] (-0.8953, -1.1821) circle (0.07);
		\draw[fill, blue, opacity=0.3] (-0.3335, 1.352) circle (0.07);
		\draw[fill, blue, opacity=0.3] (-0.4852, -1.6634) circle (0.07);
	\end{tikzpicture}
\end{center}

\paragraph{Biais évitable}
La distance entre l'espace $\Hyp$ et $f$ va diminuer si on augmente l'expressivité de $h$ et notamment en augmentant la dimension.

\paragraph{Conclusion}
Malheureusement, augmenter la dimension augmente aussi la variance. Il faut donc trouver un bon compromis sur la dimension pour réduire le biais sans trop augmenter la variance.
\begin{center}
	\begin{tikzpicture}[thick, >={latex}]
		\draw[->] (0, 0) -- (5, 0) node[below] {dimension de $\Hyp$};
		\draw[->] (0, 0) -- (0, 3.5);
		\draw[domain=0:4.2, smooth, variable=\x, blue]
			plot ({\x}, {3 - \x * (0.33 + 0.05 * \x)})
			node[right] {\footnotesize Biais};
		\draw[domain=0:4.2, smooth, variable=\x, blue]
			plot ({\x}, {0.1 + 0.08 * \x + 0.42 * exp(1.44 * \x - 4.2)})
			node[right] {\footnotesize Variance};
		\draw[domain=0:4, smooth, variable=\x, red]
			plot ({\x}, {3.1 - \x * (0.25 + 0.05 * \x) + 0.42 * exp(1.44 * \x - 4.2)});
		\draw[red] (2.85, 0.3) -- (2.85, 2.7) node[above] {\footnotesize $\trisk(h)^{min}$};
	\end{tikzpicture}
\end{center} 

\subs{Bornes de généralisation}

\paragraph{But}
Notre objectif est d'obtenir des bornes \textbf{PAC (Probably Approximately Correct)} de la forme suivante : Avec probabilité $1 - \delta$
$$ \begin{array}{lll}
	\trisk(h) & \leqslant & \erisk(h) + \gamma \\
	 & \leqslant & \erisk(h^*) + \gamma \qquad \text{(car } h = \argmin_{h_i \in \Hyp} \erisk(h_i)) \\
	 & \leqslant & (\trisk(h^*) + \gamma) + \gamma \\
	 & \leqslant & \trisk(h^*) + 2 \gamma
\end{array} $$

La théorie de la convergence uniforme nous donne des garanties pour les hypothèses $h \in \Hyp$. La question que l'on se pose est : Sous quelles conditions (sur le nombre minimum d'exemples d'entraînement requis) peut-on obtenir des bornes PAC valides ? \\
On va considérer deux situations. La première est celle où $|\Hyp| = k$ est fini. La seconde est celle où $\Hyp$ est infini.

\subsubs{Convergence uniforme - Cas fini}

On commence par rappeler le lemme suivant: \vspace{3mm}
\PROP[ (Inégalité de Hoeffding)]{
	Soit $Z_1, \dots, Z_m$ $m$ variables i.i.d suivant des loi de Bernoulli d'espérance $\phi$. On pose la variable $\hat{\phi} = \frac{1}{m} \sum_{i = 1}^m Z_i$ et on considère $\gamma > 0$. Alors : \vspace{-2mm}
	$$ \Pp(|\hat{\phi} - \phi| > \gamma) \leqslant 2 \exp(-2 \gamma^2 m) $$
	\vspace{-9mm}
}
\vspace{2mm}

On considère alors l'espace $\Hyp = \{ h_1, \dots, h_k \}$. L'inégalité de Hoeffding peut être appliqué à $\trisk(h)$ et $\erisk(h)$ avec $l(h, z_i)$ qui est une loi de Bernoulli d'espérance $\trisk(h)$. On pose donc $A_j$ l'événement $| \trisk(h_j) - \erisk(h_j) | \geqslant \gamma$. Avec l'inégalité de Hoeffding, on a : $\Pp(A_j) \leqslant 2 e^{-2 \gamma^2 m}$. Et cela donne :
$$ \begin{array}{lll}
	\Pp(\sup_{h \in \Hyp} | \trisk(h) - \erisk(h) | \geqslant \gamma )
	 & = & \Pp(A_1 \cup \dots \cup A_k) \\
	 & \leqslant & \sum_j \Pp(A_j) \\
	 & \leqslant & \sum_j 2 e^{-2 \gamma^2 m} \\
	 & \leqslant & 2k e^{-2 \gamma^2 m}
\end{array} $$

\paragraph{Borne sur $m$}
Avec l'inégalité précédente on peut essayer de trouver la valeur minimale de $m$ pour que la probabilité soit au plus $\delta$ :
$$ \begin{array}{lll}
	2k e^{-2 \gamma^2 m} \leqslant \delta
	 & \Leftrightarrow & e^{2 \gamma^2 m} \geqslant \dfrac{2k}{\delta} \\
	 & \Leftrightarrow & 2 \gamma^2 m \geqslant \ln \left( \dfrac{2k}{\delta} \right) \\
	 & \Leftrightarrow & m \geqslant \dfrac{1}{2 \gamma^2} \ln \left( \dfrac{2k}{\delta} \right)
\end{array} $$
Donc si $m \geqslant \dfrac{1}{2 \gamma^2} \ln \left( \dfrac{2k}{\delta} \right)$ alors avec probabilité $1 - \delta$, on a :
$$ \trisk(h) \leqslant \erisk(h) + \gamma $$
Mais généralement $m$ est fixé.

\paragraph{Borne sur $\gamma$}
Pour un $m$ fixé et une probabilité $\delta$ fixée, on obtient :
$$ \gamma = \sqrt{\dfrac{1}{2m} \ln \left( \dfrac{2k}{\delta} \right)} $$

\PROP[ (Borne de généralisation dans le cas fini)] {
	Avec probabilité $1 - \delta$, on a pour tout $h$ dans $\Hyp$ :
	$$ \trisk(h) \leqslant \erisk(h) + \sqrt{\dfrac{1}{2m} \ln \left( \dfrac{2k}{\delta} \right)} $$
	\vspace{-5mm}
}

\subsubs{Convergence uniforme - Cas infini}

On introduit la dimension VC (pour Vapnik-Chervonenkis) qui est une mesure de la capacité (ou complexité) de la classe des hypothèses $\Hyp$.

\DEF{
	Un ensemble de point $S$ est \textbf{pulvérisé} par $\Hyp$ si pour tout sous-ensembles $A$ de $S$, il existe une hypothèse $h \in \Hyp$ qui ne fait pas d'erreur sur $A$. Autrement dit $S$ est pulvérisé par $\Hyp$ si les éléments de $\Hyp$ permettent d'obtenir les $2^{|S|}$ dichotomies de $S$. \\
	La \textbf{dimension VC} $d_\Hyp$ d'une classe d'hypothèses $\Hyp$ est défini comme le plus grand cardinal de points que $\Hyp$ peut pulvériser.
}

\begin{center}
	\begin{tikzpicture}[thick, >={latex}]
		\draw (0, 5.5) circle (0.12);
		\draw (1, 5.5) circle (0.12);
		\draw (1, 4.5) circle (0.12);
		\draw (4, 5.5) circle (0.12);
		\draw (10, 5.5) circle (0.12);
		\draw (10, 4.5) circle (0.12);
		\draw (1, 2.5) circle (0.12);
		\draw (3, 3.5) circle (0.12);
		\draw (6, 3.5) circle (0.12);
		\draw (7, 3.5) circle (0.12);
		\draw (9, 3.5) circle (0.12);
		\draw (10, 2.5) circle (0.12);
		
		\draw[fill] (3, 5.5) circle (0.12);
		\draw[fill] (4, 4.5) circle (0.12);
		\draw[fill] (6, 5.5) circle (0.12);
		\draw[fill] (7, 5.5) circle (0.12);
		\draw[fill] (7, 4.5) circle (0.12);
		\draw[fill] (9, 5.5) circle (0.12);
		\draw[fill] (0, 3.5) circle (0.12);
		\draw[fill] (1, 3.5) circle (0.12);
		\draw[fill] (4, 3.5) circle (0.12);
		\draw[fill] (4, 2.5) circle (0.12);
		\draw[fill] (7, 2.5) circle (0.12);
		\draw[fill] (10, 3.5) circle (0.12);
		
		\draw (-1, 4) -- (11, 4);
		\draw (2, 2.2) -- (2, 5.8);
		\draw (5, 2.2) -- (5, 5.8);
		\draw (8, 2.2) -- (8, 5.8);
		
		\draw (-0.5, 5.7) -- (1.2, 4);
		\draw[->] (0.35, 4.85) -- (0.05, 4.55);
		\draw (3.3, 5.7) -- (4.2, 4.8);
		\draw[->] (3.75, 5.25) -- (3.45, 4.95);
		\draw (5.5, 5.7) -- (7.2, 4);
		\draw[->] (6.35, 4.85) -- (6.65, 5.15);
		\draw (9.5, 4.2) -- (9.5, 5.7);
		\draw[->] (9.5, 5) -- (9.05, 5);
		\draw (-0.3, 3) -- (1.3, 3); 
		\draw[->] (0.5, 3) -- (0.5, 3.45);
		\draw (3.5, 2.2) -- (3.5, 3.8);
		\draw[->] (3.5, 3) -- (3.95, 3);
		\draw (5.7, 3) -- (7.3, 3); 
		\draw[->] (6.5, 3) -- (6.5, 2.55);
		\draw (9.2, 3.7) -- (10.2, 2.7);
		\draw[->] (9.7, 3.2) -- (9.9, 3.4);
	\end{tikzpicture}
\end{center}

Avec $d_\Hyp$ on peut obtenir une borne pour $\trisk(h)$.

\PROP[ (Borne de généralisation dans le cas infini)] {
	Avec probabilité $1 - \delta$, on a pour tout $h$ dans $\Hyp$ :
	$$ \trisk(h) \leqslant \erisk(h) + \sqrt{\dfrac{d_\Hyp \left( \ln \frac{2 m}{d_\Hyp} + 1 \right) + ln \frac{4}{\delta}}{m}} $$
	\vspace{-5mm}
}

Au lieu d'utiliser la dimension VC, on peut aussi utiliser la complexité de Rademacher.

\DEF{
	La \textbf{complexité empirique de Rademacher} de $\Hyp$ est :
	$$ Rad_m(\Hyp, S) = \E \left( \sup_{h \in \Hyp} \left| \frac{1}{m} \sum_{i = 1}^m \sigma_i h(z_i) \right| \right) $$
	Où $\sigma_1, \dots, \sigma_m$ sont $m$ variables de Rademacher i.i.d avec $\Pp(\sigma_i = 1) = \Pp(\sigma_i = -1) = \frac{1}{2}$.
}

\PROP[ (Borne de convergence uniforme avec la complexité de Rademacher)] {
	Avec probabilité $1 - \delta$, on a pour tout $h$ dans $\Hyp$ : \vspace{-3mm}
	$$ \trisk(h) \leqslant \erisk(h) + 2 Rad_m(\Hyp, S) + \sqrt{\dfrac{4}{m} ln \left( \dfrac{2}{\delta} \right)} $$
	\vspace{-5mm}
}

\subsubs{Stabilité uniforme}

On veut réduire la variance sans modifier le biais. On va voir qu'avoir une faible variance est équivalent à avoir une grande stabilité. Intuitivement un algorithme est stable si pour de petits changements dans l'ensemble d'entraînement, la sortie $h$ de l'algorithme varie peu.

Dans la définition qui suit, on utilise les notations suivantes :
\begin{itemize}
	\item $S^{\setminus i} = \{ z_1, \dots, z_{i-1}, z_{i+1}, \dots, z_m \}$ pour l'ensemble $S$ sans le $i$-ème exemple.
	\item $S^{i} = \{ z_1, \dots, z_{i-1}, z_i^\prime, z_{i+1}, \dots, z_m \}$ pour l'ensemble $S$ dont le $i$-ème exemple est remplacé par une valeur aléatoire suivant la loi $\dist$.
\end{itemize}

\DEF{
	Un algorithme $L$ a une \textbf{stabilité uniforme} $\dfrac{\beta}{m}$ respectivement à une fonction de perte $l$ si :
	$$ \forall S, \, \forall i \in \{1, ..., m\} \, \sup_z \left| l(h_S, z) - l(h_{S^{\setminus i}}, z) \right| \leqslant \dfrac{\beta}{m} $$
	\vspace{-5mm}
}

Par inégalité triangulaire on obtient :
$$ \forall S, \, \forall i \in \{1, ..., m\} \, \sup_z \left| l(h_S, z) - l(h_{S^{i}}, z) \right| \leqslant 2 \dfrac{\beta}{m} $$


\PROP[ (Borne de généralisation utilisant la stabilité uniformes)] {
	Soit $S$ un ensemble d'entraînement de taille $m$. Alors pour tout algorithme $L$ de stabilité uniforme $\dfrac{\beta}{m}$ respectivement à une fonction de perte $l$ bornée par $M$, on a avec probabilité $1 - \delta$ : \vspace{-3mm}
	$$ \trisk(h_S) \leqslant \erisk(h_S) + 2 \dfrac{\beta}{m} + (4 \beta + M) \sqrt{\dfrac{\ln \frac{1}{\delta}}{2m}} $$
	\vspace{-5mm}
}

Pour la démonstration, on a besoin du théorème suivant :

\PROP[ (McDiarmid)]{
	Soit $F : \Z^m \rightarrow \R$ une fonction pour laquelle il existe des constantes $c_1, ..., c_m$ tel que \vspace{-2mm}
	$$ \sup_{S \in \Z^m, z_i^\prime \in \Z} \left| F(S) - F(S^i) \right| \leqslant c_i $$
	\vspace{-2mm} Alors \vspace{-2mm}
	$$ \Pp(F(S) - \E_S[F(S)] \geqslant \gamma) \leqslant \exp \left( \dfrac{- 2 \gamma^2}{\sum_{i = 1}^m c_i^2} \right) $$
	\vspace{-5mm}
}

\dem
Comme l'espérance est inférieure au $\sup$ d'une variable aléatoire, on a :
$$ \begin{array}{llc}
	| \trisk(h_S) - \trisk(h_{S^{\setminus i}}) | & = & \left| \E_z[l(h_S, z)] - \E_z[l(h_{S^{\setminus i}}, z)] \right| \\
	 & \leqslant & \dfrac{\beta}{m}
\end{array} $$
On utilise alors l'inégalité triangulaire pour déduire :
$$ \begin{array}{llc}
| \trisk(h_S) - \trisk(h_{S^{i}}) | & = & | \trisk(h_S) - \trisk(h_{S^{\setminus i}}) | + | \trisk(h_{S^i}) - \trisk(h_{S^{\setminus i}}) | \\
& \leqslant & 2 \dfrac{\beta}{m}
\end{array} $$
On passe ensuite au risque empirique :
$$ \begin{array}{lll}
	| \erisk(h_S) - \erisk(h_{S^{\setminus i}}) |
	 & = & \left| \displaystyle \dfrac{1}{m} \sum_{j \neq i} \left( l(h_S, z_j) - l(h_{S^{\setminus i}}, z_j) \right) + \dfrac{1}{m} l(h_S, z_i) \right| \\
	 & \leqslant & \displaystyle \dfrac{1}{m} \sum_{j \neq i} \left| l(h_S, z_j) - l(h_{S^{\setminus i}}, z_j) \right| + \dfrac{1}{m} | l(h_S, z_i) | \\
	 & \leqslant & \dfrac{\beta}{m} + \frac{M}{m}
\end{array}$$
Toujours avec l'inégalité triangulaire, on obtient ensuite :
$$ | \erisk(h_S) - \erisk(h_{S^{i}}) | \leqslant 2 \dfrac{\beta}{m} + 2 \dfrac{M}{m} $$
Cependant on peut faire un peu mieux :
$$ \begin{array}{lll}
	| \erisk(h_S) - \erisk(h_{S^{i}}) |
	& \leqslant & \displaystyle \dfrac{1}{m} \sum_{j \neq i} \left| l(h_S, z_j) - l(h_{S^{ i}}, z_j) \right| + \dfrac{1}{m} | l(h_S, z_i) - l(h_{S^i}, z_i^\prime) | \\
	& \leqslant & 2 \dfrac{\beta}{m} + \frac{M}{m}
\end{array}$$
On considère alors la fonction $F = \trisk - \erisk$ pour utiliser le théorème de McDiarmid :
$$ \begin{array}{lll}
	| F(S) - F(S^i) |
	& = & | (\trisk(h_S) - \erisk(h_S)) - (\trisk(h_{S^i}) - \erisk(h_{S^i}) | \\
	& \leqslant & | \trisk(h_S) - \trisk(h_{S^i}) | + | \erisk(h_S) - \erisk(h_{S^i}) | \\
	& \leqslant & 4 \dfrac{\beta}{m} + \dfrac{M}{m} = c_i
\end{array} $$
D'après le théorème de McDiarmid on a donc :
$$ \Pp \left(F(S) - \E_S[F(S)] \geqslant \gamma \right) \leqslant \exp \left( \dfrac{- 2 \gamma^2}{\sum_{i = 1}^m c_i^2} \right) $$
Il nous faut désormais calculer l'espérance de $F(S)$.
$$ \begin{array}{lll}
	\E_S[\trisk(h_S) - \erisk(h_S)]
	& = & \E_S[\trisk(h_S)] - \E_S[\erisk(h_S)] \\
	& = &\displaystyle \E_{S, z \sim \dist}[l(h_S, z)] - \dfrac{1}{m} \sum_{j = 1}^m \E_S[l(h_S, z_j)] \\
	& = & \displaystyle \E_{S, z \sim \dist}[l(h_S, z)] - \dfrac{1}{m} \sum_{j = 1}^m \E_{S, z_j^\prime \sim \dist}[l(h_{S^j}, z_j^\prime)] \\
	& = & \E_{S, z_j^\prime \sim \dist}[l(h_S, z_j^\prime)-l(h_{S^j}, z_j^\prime)] \\ \vspace{1mm}
	& \leqslant & \displaystyle \E_{S, z_j^\prime \sim \dist} \left[ | l(h_S, z_j^\prime)-l(h_{S^j}, z_j^\prime) | \right] \\
	& \leqslant & \displaystyle \E_{S, z_j^\prime \sim \dist} \left[ | l(h_S, z_j^\prime)-l(h_{S^{\setminus j}}, z_j^\prime) | \right] + \E_{S, z_j^\prime \sim \dist} \left[ | l(h_{S^{\setminus j}}, z_j^\prime)-l(h_{S^j}, z_j^\prime) | \right] \\
	& \leqslant & 2 \dfrac{\beta}{m}
\end{array} $$
Cela nous donne finalement :
$$ \Pp \left( \trisk - \erisk \geqslant \gamma + 2 \dfrac{\beta}{m} \right) \leqslant \exp \left( \dfrac{- 2 m \gamma^2}{(4 \beta + M)^2} \right) $$
Et en prenant $\gamma$ tel que la partie de droite soit $1 - \delta$ on obtient bien :
$$ \trisk(h_S) \leqslant \erisk(h_S) + 2 \dfrac{\beta}{m} + (4 \beta + M) \sqrt{\dfrac{\ln \frac{1}{\delta}}{2m}} $$
\findem

Maintenant il nous faut savoir comment trouver $\beta$.

\PROP[ (Stabilité avec régularisation)]{
	Si notre apprentissage est de la forme :
	$$ \min_{h \in \Hyp} \dfrac{1}{m} \sum_i l(h, z_i) + \lambda \| h \|^2 $$
	Alors si $l$ est $\sigma$-admissible, c'est à dire que $x \mapsto l(h(x), y)$ est $\sigma$-lipschitzienne pour tout $h$ et tout $y$, on a :
	$$ \beta \leqslant \dfrac{\sigma^2}{2 \lambda} $$
	\vspace{-5mm}
}

\subsubs{Robustesse algorithmique}

\DEF{
	Un algorithme $L$ est $\bm{(K, \epsilon(.))}$\textbf{-robuste} pour $K \in \N$ et $\epsilon : \mathcal{P}(\Z) \rightarrow \R$ si pour tout $S \in \mathcal{P}(\Z)$, $\Z$ peut être partitionné en $K$ sous-ensembles $\{ C_i \}_{i = 1}^K$ tel que :
	$$ \forall i \in [K], \, \forall z, z' \in C_i, \, z \in S \Rightarrow |l(h_S, z) - l(h_S, z')| \leqslant \epsilon(S) $$
	\vspace{-5mm}
}

\PROP[ (borne de robustesse)]{
	Si un algorithme est $(K, \epsilon(.))$-robuste, alors avec probabilité $1 - \delta$ :
	$$ \trisk(h_S) \leqslant \erisk(h_S) + \epsilon(S) + \sqrt{\dfrac{2K \ln 2 + 2 \ln(1/\delta)}{m}} $$
	\vspace{-5mm}
}

\subs{Choix du modèle}

\renewcommand{\trisk}{\mathcal{R}}
\renewcommand{\erisk}{\hat{\mathcal{R}}}

Les bornes de généralisation PAC ne sont pas vraiment utilisables car elles sont vraiment pessimistes. Une idée peut être la validation croisée qui permet de mesurer la généralisation d'un modèle après entraînement.

\begin{center}
	\begin{algorithm}[H]
		\KwIn{Un algorithme d'apprentissage $L$ et un ensemble d'apprentissage $S$}
		\KwOut{une estimation $\erisk_2(h)$ de $\trisk(h)$}
		\vspace{3mm}
		Séparer $S$ en $k$ sous-ensembles $S_1, \dots, S_k$\;
		\For{$i=1$ à $k$}{
			Exécuter $L$ sur $S \setminus S_i$ pour obtenir un classifieur $h_i$;
		}
		Déduire l'estimation du risque $\erisk_2(h) = \frac{1}{k} \sum_{i = 1}^k \erisk_2(h_i)$ où $\erisk_2(h_i)$ est l'erreur de $h_i$ sur $S_i$\;
		\vspace{3mm}
		\caption{Algorithme de $k$-validation croisée}
	\end{algorithm}
\end{center}

Avec cet algorithme on peut alors comparer différent algorithmes d'apprentissage.

\exe
\begin{center}
	\begin{tikzpicture}[thick, scale=0.6]
		\draw[domain=0:10, smooth, variable=\x, blue, dashed]
			plot ({\x}, {-0.55 + 1.3 * \x * (1 - 0.1 * \x)});
		\draw[domain=0:10, smooth, variable=\x, red]
			plot ({\x}, {0.895863 + 0.14997759 * \x)});
		\draw[circle, fill] (0.1, -0.6349920559756366) circle(0.15);
		\draw[circle, fill] (1.0, 0.30147401369268423) circle(0.15);
		\draw[circle, fill] (1.5, 1.3173171840454918) circle(0.15);
		\draw[circle, fill] (2.3, 2.2037856155666598) circle(0.15);
		\draw[circle, fill] (4.8, 2.53199171235714) circle(0.15);
		\draw[circle, fill] (5.1, 3.0593432681003296) circle(0.15);
		\draw[circle, fill] (6.6, 2.7147727490453817) circle(0.15);
		\draw[circle, fill] (7.4, 2.377546686262505) circle(0.15);
		\draw[circle, fill] (8.2, 1.5487114646843132) circle(0.15);
		\draw[circle, fill] (9.2, 0.46764396433128286) circle(0.15);
		\node at (5, -1) {underfitting};
		
		\draw[domain=0:10, smooth, variable=\x, blue, dashed]
			plot ({\x+14}, {-0.55 + 1.3 * \x * (1 - 0.1 * \x)});
		\draw[domain=0:2, smooth, variable=\x, red]
			plot ({\x+14}, {-1.41950554e+00 + \x * (1.00967400e+01 + \x * (-2.53146666e+01 +
				\x * (2.96771161e+01 + \x * (-1.74688908e+01 + \x * (5.69941439e+00 +
				\x * (-1.08044823e+00 + \x * (1.18525383e-01 + \x * (-6.98156850e-03 +
				\x * 0.000170922685)))))))});
		\draw[domain=-1:2, smooth, variable=\x, red]
			plot ({\x+17}, {1.42484270e+00 + \x * (-1.40170537e+00 + \x * (3.40897007e-01 +
			\x * (1.03657493e+00 + \x * (-1.83881811e-01 + \x * (-1.59051020e-01 +
			\x * (3.68822015e-02 + \x * (6.34668910e-03 + \x * (-2.36665600e-03 +
			\x * 1.70922685e-04)))))))});
		\draw[domain=-1:2, smooth, variable=\x, red]
		plot ({\x+20}, {3.33547340e+00 + \x * (-7.96494283e-01 + \x * (-8.55389836e-01 +
			\x * (6.85820570e-01 + \x * (2.21442569e-01 + \x * (-1.29594094e-01 +
			\x * (-3.85819884e-02 + \x * (4.92589522e-03 + \x * (2.24825651e-03 +
			\x * 1.70922685e-04)))))))});
		\draw[domain=-1:0.36, smooth, variable=\x, red]
			plot ({\x+23}, {-1.02813387e+00 + \x * (2.33112518e+00 + \x * (1.98927127e+01 +
				\x * (2.58711983e+01 + \x * (1.57048589e+01 + \x * (5.25072508e+00 +
				\x * (1.01907510e+00 + \x * (1.14263001e-01 + \x * (6.86316901e-03 +
				\x * 1.70922685e-04)))))))});
		\draw[circle, fill] (14.1, -0.6349920559756366) circle(0.15);
		\draw[circle, fill] (15.0, 0.30147401369268423) circle(0.15);
		\draw[circle, fill] (15.5, 1.3173171840454918) circle(0.15);
		\draw[circle, fill] (16.3, 2.2037856155666598) circle(0.15);
		\draw[circle, fill] (18.8, 2.53199171235714) circle(0.15);
		\draw[circle, fill] (19.1, 3.0593432681003296) circle(0.15);
		\draw[circle, fill] (20.6, 2.7147727490453817) circle(0.15);
		\draw[circle, fill] (21.4, 2.377546686262505) circle(0.15);
		\draw[circle, fill] (22.2, 1.5487114646843132) circle(0.15);
		\draw[circle, fill] (23.2, 0.46764396433128286) circle(0.15);
		\node at (19, -1) {overfitting};
		
		\draw[domain=0:10, smooth, variable=\x, blue, dashed]
			plot ({\x+7}, {-6.55 + 1.3 * \x * (1 - 0.1 * \x)});
		\draw[domain=0:10, smooth, variable=\x, red]
			plot ({\x+7}, {-6.76178277 + \x * (1.48142299 - 0.14599194 * \x)});
		\draw[circle, fill] (7.1, -6.634992055975637) circle(0.15);
		\draw[circle, fill] (8.0, -5.698525986307316) circle(0.15);
		\draw[circle, fill] (8.5, -4.682682815954508) circle(0.15);
		\draw[circle, fill] (9.3, -3.7962143844333402) circle(0.15);
		\draw[circle, fill] (11.8, -3.46800828764286) circle(0.15);
		\draw[circle, fill] (12.1, -2.9406567318996704) circle(0.15);
		\draw[circle, fill] (13.6, -3.2852272509546183) circle(0.15);
		\draw[circle, fill] (14.4, -3.622453313737495) circle(0.15);
		\draw[circle, fill] (15.2, -4.451288535315687) circle(0.15);
		\draw[circle, fill] (16.2, -5.532356035668717) circle(0.15);
		\node at (12, -7) {Bonne généralisation};		
	\end{tikzpicture}
\end{center}

On cherche un modèle sous forme de polynôme.
$$ h_{\theta^{(1)}}(x) = \theta^{(1)}_0 + \theta^{(1)}_1 x $$
$$ h_{\theta^{(2)}}(x) = \theta^{(2)}_0 + \theta^{(2)}_1 x + \theta^{(2)}_2 x^2 $$
$$ \dots $$
$$ h_{\theta^{(10)}}(x) = \theta^{(10)}_0 + \theta^{(10)}_1 x + \dots + \theta^{(10)}_{10} x^{10} $$
On souhaite savoir quel degré $d^* \in \{ 1, ..., 10 \}$ on doit choisir. Ici $\theta$ est l'ensemble de paramètres à apprendre et $d^*$ peut être vu comme un paramètre extra à régler. \\

On peut alors séparer notre ensemble d'entraînement en deux sous-ensembles $S = Train \cup Test$. Le degré optimal peut ensuite être obtenu en apprenant les paramètres $\theta^{1}, \dots, \theta^{(10)}$ grâce au sous-ensemble $Train$ puis en sélectionnant le modèle avec la plus petite erreur sur le sous-ensemble $Test$.

\paragraph{Cependant !} On ne peut pas réutiliser l'ensemble $Test$ pour estimer le vrai risque. En effet cela ne serait pas équitable car on a choisi le paramètre $d^*$ de sorte à minimiser l'erreur sur $Test$. L'erreur de généralisation obtenue sur cette ensemble est donc bien trop optimiste.	On doit donc finalement séparé notre ensemble d'entraînement en trois sous-ensembles :
\begin{itemize}
	\item \textbf{L'ensemble d'entraînement} pour apprendre les paramètres $\theta^{(1)}, \dots, \theta^{(10)}$.
	\item \textbf{L'ensemble de validation} pour choisir le meilleur paramètre $d^*$.
	\item \textbf{L'ensemble de teste} pour estimer l'erreur de généralisation de $h_{\theta^{(d^*)}}$.
\end{itemize}
Si $\erisk(h)$ est l'erreur obtenue sur l'ensemble d'entraînement alors on peut voir $\erisk(h)$ comme le biais de $h$. De même si $\vrisk(h)$ est l'erreur obtenue sur l'ensemble de validation alors on peut voir $V = \vrisk(h) - \erisk(h)$ comme la variance de $h$.

\exe
On obtient les résultats suivants :
\begin{itemize}
	\item Erreur d'entraînement : $1\%$
	\item Erreur de validation : $11\%$
\end{itemize}
Cela veut dire que la variance est bien plus élevée que le biais. On peut donc corriger cela en entraînant l'algorithme sur une plus grand ensemble d'entraînement. \\
En revanche si on obtient les résultats suivants :
\begin{itemize}
	\item Erreur d'entraînement : $15\%$
	\item Erreur de validation : $16\%$
\end{itemize}
Cela veut dire que c'est le biais qui est élevé par rapport à la variance. Il faut donc augmenter l'expressivité de notre modèle.
	\chapter{Régression Linéaire/Polynomiale/Logistique}

\myminitoc

SOON !!!

\sect{Introduction}

On va considérer des modèles $h$ sous forme linéaire.
$$ h_\theta(x) = \sum_{i = 0}^n \theta_i x_i = \theta^\trans x $$
On posera toujours $x_0 = 1$ pour avoir une ordonnée à l'origine. Le but sera alors de trouver $\theta \in \R^{n+1}$ pour que $h$ fasse des prédictions précises.

\begin{center}
	\begin{tikzpicture}[thick, >={latex}]
		\draw[->, blue] (0, 0) -- (6, 0);
		\draw[->, blue] (0, 0) -- (0, 4);
		\draw[fill] (0.357143, 0.395123) circle (0.10);
		\draw[fill] (1.146921, 1.641947) circle (0.10);
		\draw[fill] (1.740568, 1.906676) circle (0.10);
		\draw[fill] (1.835173, 1.007108) circle (0.10);
		\draw[fill] (2.915078, 2.585541) circle (0.10);
		\draw[fill] (3.815745, 2.650688) circle (0.10);
		\draw[fill] (4.439558, 2.812134) circle (0.10);
		\draw[domain=0:5.000000, smooth, variable=\x, red] 
			plot ({\x}, {0.568668 + \x * 0.554981});
		\draw [decorate, decoration={brace,amplitude=2pt}, xshift=-0.1cm, greenTikz]
			(0, 0) -- node [left] {$\theta_0$} (0, 0.568);
		\draw[dashed] (1.835, 1) -- (1.835, 1.59);
		\draw [decorate, decoration={brace,amplitude=2pt}, xshift=0.2cm, greenTikz]
			(1.835, 1.59) -- node [right] {$h_\theta(x) - y$} (1.835, 1);
	\end{tikzpicture}
\end{center}

On se posera alors le \textbf{problème des moindres carrées} :
$$ \min_\theta J(\theta) = \min_\theta \dfrac{1}{2m} \sum_{i = 1}^m \left( h_\theta(x^{(i)}) - y^{(i)} \right)^2 $$
Il y a plusieurs méthodes pour minimiser $J(\theta)$ :
\begin{itemize}
	\item \textbf{Descente de gradient par batchs}
	\item \textbf{Descente de gradient stochastique}
	\item \textbf{Descente de gradient par mini-batchs}
	\item \textbf{Solution de forme fermée}
\end{itemize}

\sect{Régression linéaire et polynomiale}

\subs{Descente de gradient}

\subsubs{Descente de gradient par batchs}

\paragraph{Idée basique}
Si $J(\theta)$ est différentiable, l'idée est la suivante :
\begin{itemize}
	\item Initialiser $\theta$ avec la valeur 0 ou par un vecteur aléatoire.
	\item Mettre à jour $\theta$ de manière à réduire $J(\theta)$ après avoir calculer les dérivées partielles de $J(\theta)$.
	\item Puis répéter ce processus jusqu'à convergence vers un minimum de $J(\theta)$.
\end{itemize}
La formule de mise à jour de $\theta$ est la suivant :
$$ \theta_i \gets \theta_i - \alpha \dfrac{\partial}{\partial \theta_i} J(\theta) $$
Où $\alpha$ est une constante qui s'appelle le \textbf{taux d'apprentissage} et qui permet de contrôler la grandeur des pas que l'on fait. \\
En utilisant la notation du gradient : $ \nabla_\theta J = \begin{bmatrix}
\frac{\partial}{\partial \theta_0} \\ \vdots \\ \frac{\partial}{\partial \theta_n}
\end{bmatrix} $, on peut réécrire la mise à jour comme cela :
\begin{center}
	\boldmath \fbox{$ \displaystyle \theta \gets \theta - \alpha \nabla_\theta J$}
\end{center}

\paragraph{Mise à jour du $i$-ème paramètre}
On se place dans le cas où $m = 1$. C'est à dire :
$$ J(\theta) = \dfrac{1}{2} (h_\theta(x) - y)^2 $$
On calcule alors la dérivée partielle selon $\theta_i$ :
$$ \begin{array}{lll}
	\dfrac{\partial}{\partial \theta_i} J(\theta)
	& = & \dfrac{\partial}{\partial \theta_i} \dfrac{1}{2} (h_\theta(x) - y)^2 \\
	& = & 2 \times \dfrac{1}{2} (h_\theta(x) - y) \times \dfrac{\partial}{\partial \theta_i}  (h_\theta(x) - y) \\
	& = & (h_\theta(x) - y) \times \dfrac{\partial}{\partial \theta_i}  (\theta_0 x_0 + ... + \theta_i x_i + ... + \theta_n x_n - y) \\
	& = & (h_\theta(x) - y) x_i
\end{array} $$
Ce qui nous permet d'avoir : $ \theta_i \gets \theta_i - \alpha (h_\theta(x) - y) x_i $. \\
Dans le cas générale on obtient alors :
\begin{center}
	\boldmath \fbox{$ \displaystyle \theta_i \gets \theta_i - \alpha \dfrac{1}{m} \sum_{j = 1}^m \left( h_\theta(x^{(j)}) - y^{(j)} \right) x_i^{(j)}$}
\end{center}
A noter qu'on parle de batch car à chaque descente de gradient on regarde l'ensemble d'entraînement en entier.

\REM{
	Avec des initialisation légèrement différentes, on peut converger vers des minimums locaux complètement différents
	\begin{center}
		\includegraphics[scale=1.6]{grad1.png}
		\includegraphics[scale=1.6]{grad2.png}
	\end{center}
}

\paragraph{Choix de $\alpha$}
Il est important de bien choisir $\alpha$, car si sa valeur est trop élevé il est possible que $J(\theta)$ croisse. Une valeur faible de $\alpha$ est donc préférable mais il faut savoir que plus sa valeur est faible, plus le temps de convergence sera long ...
\begin{center}
	\begin{tikzpicture}[thick, >={latex}, scale=0.9]
		\draw[domain=1.181978:6.956124, smooth, variable=\x, blue] 
			plot ({\x}, {0.5 * (\x - 4)^2});
		\draw[->, greenTikz]
			(2.000000, 2.000000) -- node[right] {\footnotesize petit $\alpha$} (3.500000, 0.125000);
		\draw[->, red]
			(2.000000, 2.000000) -- node[below] {\footnotesize grand $\alpha$} (6.160000, 2.332800);
		\draw[->, greenTikz] (3.500000, 0.125000) -- (3.875000, 0.007812);
		\draw[->, red] (6.160000, 2.332800) -- (1.667200, 2.720978);
		\draw[->, greenTikz] (3.875000, 0.007812) -- (3.968750, 0.000488);
		\draw[->, red] (1.667200, 2.720978) -- (6.519424, 3.173749);
		\draw[fill] (2, 2) circle (0.1);
	\end{tikzpicture}
\end{center}

\paragraph{Pour}
\begin{itemize}
	\item Peu de mises à jour sont nécessaires car le gradient est stable.
	\item La séparation en somme de la mise à jour permet d'utiliser des algorithmes parallèles
\end{itemize}
\paragraph{Contre}
\begin{itemize}
	\item La stabilité du gradient peu conduire prématurément vers un minimum local pas très optimal.
	\item La mise à jour peut prendre beaucoup de temps pour de grandes bases de données.
\end{itemize}

\subsubs{Descente de gradient stochastique}

Si la base de donnée est grande, disons 1 million d'exemples alors à chaque itération de le descente de gradient par batchs, on doit réaliser une somme d'erreurs sur un million d'exemples. Pour plus de rapidité il nous faut donc penser à un autre algorithme où l'on fait des mises à jour de $\theta$ plus régulièrement :

\begin{center}
	\begin{algorithm}[H]
		Initialisation de $\theta$\;
		\Repeat{convergence de $J(\theta)$}{
			\For{$j=1$ à $m$}{
				$\forall i, \; \theta_i \gets \theta_i - \alpha \left( h_\theta(x^{(j)}) - y^{(j)} \right) x_i^{(j)} $
			}
		}
		\caption{Descente de gradient stochastique}
	\end{algorithm}
\end{center}

\paragraph{Pour}
\begin{itemize}
	\item La fréquence élevé des mises à jour qui peut donner un apprentissage plus rapide.
	\item Ces mise à jours un peu "chaotiques" et instables peuvent empêcher de converger vers des minimas locaux.
	\item Cet algorithme permet aussi d'avoir un acquisition de données en ligne.
\end{itemize}
\paragraph{Contre}
L'algorithme ne converge pas vers le minimum globale mais il a tendance à rester autour.

\paragraph{Descente de gradient par mini-batchs}
Pour gagner en robustesse et garder l'efficacité de la descente stochastique, on peut séparer l'ensemble d'entraînement en de petits sous-ensembles (des petits batchs). Puis on reprend la descente stochastique mais au lieu de mettre à jour $\theta$ en s'appuyant sur un seul exemple à la fois, on le met à jour en fonction de l'erreur sur un mini-batch.

\subs{Solution de forme fermée}

On pose les vecteurs et matrices suivantes :
$$ x^{(j)} = \begin{pmatrix} x_0^{(j)} \\ \vdots \\ x_n^{(j)} \end{pmatrix} \qquad
X = \begin{pmatrix} {x^{(1)}}^\trans \\ \vdots \\ {x^{(m)}}^\trans \end{pmatrix} \qquad 
\theta = \begin{pmatrix} \theta_0 \\ \vdots \\ \theta_n \end{pmatrix} \qquad
y = \begin{pmatrix} y^{(1)} \\ \vdots \\ y^{(m)} \end{pmatrix} $$
Cela nous permet de réécrire $J(\theta)$.
$$ J(\theta) = \dfrac{1}{2m} (X\theta - y)^\trans (X\theta - y) $$
En un minimum de $J$, le gradient est nul. On cherche donc à résoudre :
$$ \nabla_\theta \dfrac{1}{2} (X\theta - y)^\trans (X\theta - y) = 0 $$
\newpage

$$ \begin{array}{lll}
\nabla_\theta \dfrac{1}{2} (X\theta - y)^\trans (X\theta - y)
& = & \dfrac{1}{2} \nabla_\theta \left( \theta^\trans X^\trans X \theta - \theta^\trans X^\trans y - y^\trans X \theta + y^\trans y \right) \\ \\
& = & \dfrac{1}{2} \left[ \nabla_\theta \left(\theta^\trans X^\trans X \theta \right) - 2 \nabla_\theta \left( y^\trans X \theta \right) \right] \\ \\
& = & X^\trans X \theta - X^\trans y
\end{array} $$

\PROP[ (Solution de forme fermée)]{
	Ainsi, si $X^\trans X$ est inversible, l'expression de $\theta$ suivante est optimale :
	\vspace{-1mm}
	$$\theta = (X^\trans X)^{-1} X^\trans y$$
	En revanche si $X^\trans X$ n'est pas inversible, il est possible que des features soient redondants. Appliqué l'algorithme PCA peut alors être une solution.
}

\subs{Régression polynomiale}

Dans le cas d'une régression polynomiale, on $h_\theta(x) = \theta_0 + \theta_1 x + \theta_2 x^2 + ... + \theta_n x^n$. La fonction est toujours linéaire selon $\theta$. L'idée est alors de se placer en dimension $n+1$, en posant $x_i = x^i$.

\sect{Interprétation probabiliste de la régression}

On peut se poser la question : Pourquoi les moindres carrées ? Pourquoi ne pas minimiser la valeur absolue ou encore la puissance 4 ?

Supposons que l'erreur sur $y^{(i)}$ suit une loi gaussienne. Ceci est justifié par le théorème centrale limite. On a donc :
$$ y^{(i)} = \theta^\trans x^{(i)} + \epsilon^{(i)} $$
Où $\epsilon^{(i)} \sim \mathcal{N}(0, \sigma)$. \\
On va ensuite chercher à maximiser la vraisemblance de $y$.

\DEF{
	Soit $X_1, ..., X_m$ des variables aléatoires i.i.d de densité $f(x|\theta)$ où $\theta$ est un paramètre de la loi. Pour un échantillon $X_1 = x_1, ..., X_n = x_n$ donné, la \textbf{vraisemblance} est : \vspace{-3mm}
	$$ L(\theta) = f(x_1, ..., x_n) = \prod_{i=1}^m f(x_i|\theta) $$
	\vspace{-7mm}
}

On a donc dans notre cas :
$$ L(\theta) = \Pp(y | X, \theta) = \prod_{i=1}^m \Pp(y^{(i)} | x^{(i)}, \theta) = \prod_{i=1}^m \dfrac{1}{\sqrt{2 \pi} \sigma} \exp \left( - \dfrac{(y^{(i)} - \theta^\trans x^{(i)})^2}{2 \sigma^2} \right) $$
Une bonne manière d'étudier la vraisemblance est de la passer au logarithme car elle est souvent log-concave. La log-vraisemblance est notée $l(\theta) = \ln L(\theta)$. Voici alors ce qu'on obtient si on simplifie l'expression de la log-vraisemblance :

$$ \begin{array}{lll}
	l(\theta)
	& = & \displaystyle \sum_{i = 1}^m \ln \left( \dfrac{1}{\sqrt{2 \pi} \sigma} \exp \left( - \dfrac{(y^{(i)} - \theta^\trans x^{(i)})^2}{2 \sigma^2} \right) \right) \\
	& = & \displaystyle \sum_{i = 1}^m \ln \dfrac{1}{\sqrt{2 \pi} \sigma} + \sum_{i = 1}^m \left( - \dfrac{(y^{(i)} - \theta^\trans x^{(i)})^2}{2 \sigma^2} \right) \\
	& = & \displaystyle m \ln \dfrac{1}{\sqrt{2 \pi} \sigma} - \sum_{i = 1}^m \dfrac{(y^{(i)} - \theta^\trans x^{(i)})^2}{2 \sigma^2}
\end{array} $$
Ainsi maximiser $l(\theta)$ revient à minimiser $\displaystyle J(\theta) = \sum_{i = 1}^m \dfrac{(y^{(i)} - \theta^\trans x^{(i)})^2}{2}$ \\
Donc lorsque l'on utilise l'algorithme des moindres carrées, on maximise simplement la vraisemblance en assumant que les erreurs sur les $y^{(i)}$ sont i.i.d selon une loi normale.

\sect{Régression régularisée}

Comme on l'a vu à la fin du chapitre précédent, il y a des risques d'overfitting. Une solution est la régularisation. On parle de régularisation \textbf{ridge} pour la norme $l_2$ et de régularisation \textbf{LASSO} pour la norme $l_1$.

\begin{center}
	\begin{tikzpicture}[thick, >={latex}]
	\draw[domain=0:360, smooth, variable=\t, very thick, fill=greenTikz, fill opacity=0.6]
	plot ({1.800000 + 3.035357 * cos(\t) + 0.443197 * sin(\t)}, {1.700000 + -1.107992 * cos(\t) + 1.214143 * sin(\t)});
	\draw[domain=0:360, smooth, variable=\t, very thick] 
	plot ({1.800000 + 0.758839 * cos(\t) + 0.110799 * sin(\t)}, {1.700000 + -0.276998 * cos(\t) + 0.303536 * sin(\t)});
	\draw[domain=0:360, smooth, variable=\t, very thick] 
	plot ({1.800000 + 1.517678 * cos(\t) + 0.221598 * sin(\t)}, {1.700000 + -0.553996 * cos(\t) + 0.607071 * sin(\t)});
	\draw[domain=0:360, smooth, variable=\t, very thick] 
	plot ({1.800000 + 2.276518 * cos(\t) + 0.332398 * sin(\t)}, {1.700000 + -0.830994 * cos(\t) + 0.910607 * sin(\t)});
	
	\draw[domain=0:360, smooth, variable=\t, very thick, fill=greenTikz, fill opacity=0.6]
	plot ({10.800000 + 3.289353 * cos(\t) + 0.480283 * sin(\t)}, {1.700000 + -1.200708 * cos(\t) + 1.315741 * sin(\t)});
	\draw[domain=0:360, smooth, variable=\t, very thick] 
	plot ({10.800000 + 0.822338 * cos(\t) + 0.120071 * sin(\t)}, {1.700000 + -0.300177 * cos(\t) + 0.328935 * sin(\t)});
	\draw[domain=0:360, smooth, variable=\t, very thick] 
	plot ({10.800000 + 1.644677 * cos(\t) + 0.240142 * sin(\t)}, {1.700000 + -0.600354 * cos(\t) + 0.657871 * sin(\t)});
	\draw[domain=0:360, smooth, variable=\t, very thick] 
	plot ({10.800000 + 2.467015 * cos(\t) + 0.360212 * sin(\t)}, {1.700000 + -0.900531 * cos(\t) + 0.986806 * sin(\t)});
	
	\draw[->] (0, -1.5) -- (0, 3.8);
	\draw[->] (-2, 0) -- (5, 0);
	\draw[->] (9, -1.5) -- (9, 3.8);
	\draw[->] (7, 0) -- (14, 0);
	\draw[very thick, fill=purple] (0, 0) circle (1);
	\draw[very thick, fill=purple] (9, 1) -- (8, 0) -- (9, -1) -- (10, 0) -- cycle;
	\fill[redLight] (9, 1) circle (0.16);
	\fill[redLight] (0.342898, 0.939373) circle (0.16);
	\node[red] at (4, 3.5) {Ridge};
	\node[red] at (13, 3.5) {LASSO};
	\end{tikzpicture}
\end{center}

\PROP[ (Forme fermée pour Ridge)]{
	Pour le régression ridge, la solution de forme fermée est la suivante :
	$$\theta = (X^\trans X + \lambda I_{n+1})^{-1} X^\trans y $$
	\vspace{-8mm}
}

\PROP[ (Stabilité uniforme pour Ridge)]{
	Avec la régression ridge, on obtient la borne de généralisation suivante :
	$$ \trisk(h_\theta) \leqslant \erisk(h_\theta) + \dfrac{4 B^2}{\lambda m} + \left( \dfrac{8 B^2}{\lambda} + 2B \right) \sqrt{\dfrac{\ln 1/\delta}{2m}} $$
	Où $\Y$ est bornée et $\Y = [0, B]$.
}

En revanche pour la régularisation LASSO, on ne dispose pas de résultats précis, mais on sait que l'utilisation de cette régularisation conduit à une diminution du nombre de paramètres non nuls. Ainsi LASSO permet d'avoir le vecteur $\bm{\theta}$ \textbf{creux}.

\sect{Machine à vecteur de support (SVR)}

Jusque là nous utilisé la méthode des moindres carrés. Il y a aussi la \textbf{perte $\bm{\epsilon}$-sensible} :
$$ \min_{\theta, \xi, \xi^*} \frac{1}{2} \| \theta \|_2^2 + C \sum_{i = 1}^m \left( \xi_i + \xi_i^* \right) $$
\vspace{-2mm}
$$ \text{s.t. } \left\{ \begin{array}{l}
	y^{(i)} - \theta^\trans x^{(i)} \leqslant \epsilon + \xi_i \\
	\theta^\trans x^{(i)} - y^{(i)} \leqslant \epsilon + \xi_i^* \\
	\xi_i, \xi_i^* \geqslant 0
\end{array} \right. $$

Le lagrangien de ce problème est alors le suivant :
$$ L(\theta, \xi, \xi^*, \alpha, \alpha^*, \beta, \beta^*) = \dfrac{1}{2} \| \theta \|^2 + C \mathbbm{1}_m^\trans \left( \xi + \xi^* \right) + \alpha^\trans \left( y - X \theta - \epsilon \mathbbm{1}_m - \xi \right) + {\alpha^*}^\trans \left( X \theta - y - \epsilon \mathbbm{1}_m - \xi \right) - \beta^\trans \xi - {\beta^*}^\trans \xi^*$$
On rappelle que la fonction objective du dual est :
$$ g(\alpha, \alpha^*, \beta, \beta^*) = \inf_{\theta, \xi, \xi^*} L(\theta, \xi, \xi^*, \alpha, \alpha^*, \beta, \beta^*) $$
On cherche maintenant ce que vaut $\theta$ dans l'expression de $g$.
$$ \dfrac{\partial L}{\partial \theta} = \theta - X^\trans \alpha + X^\trans \alpha^* $$
$$ \dfrac{\partial L}{\partial \xi} = C \mathbbm{1}_m - \alpha - \alpha^* - \beta \qquad \qquad
\dfrac{\partial L}{\partial \xi^*} = C \mathbbm{1}_m - \alpha - \alpha^* - \beta^* $$
D'où $\theta = X^\trans (\alpha - \alpha*)$ et $\beta = \beta^* = C \mathbbm{1} - (\alpha + \alpha^*)$. On obtient alors :
$$ g(\alpha, \alpha^*) = -\dfrac{1}{2} (\alpha - \alpha^*)^\trans X X^\trans (\alpha - \alpha^*) + (\alpha - \alpha^*)^\trans y - \epsilon (\alpha + \alpha^*)^\trans \mathbbm{1}_m $$
Et on a la condition $\alpha + \alpha^* \leqslant C \mathbbm{1}_m$ car $\beta$ et $\beta^*$ sont positifs. \\ Soit alors $(\alpha, \alpha^*)$ une solution. Supposons $\alpha \geqslant \alpha^*$. On pose $\nu = \alpha - \alpha^*$. On a alors $\nu \leqslant \alpha + \alpha^*$ et $\nu - 0 = \alpha - \alpha^*$. Donc $(\nu, 0)$ est aussi une solution, et $g(\nu, 0) \geqslant g(\alpha, \alpha^*)$. \\
On obtient finalement le problème dual suivant :
\begin{center}
	\fbox{$ \displaystyle \min_{|\nu_i| \leqslant C} \dfrac{1}{2} \left\| X^\trans \nu \right\|_2^2 - \nu^\trans y + \epsilon \| \nu \|_1 $}
\end{center}
Dans ce cas, on a $\theta = X^\trans \nu$, et $h_\nu(x) = \theta^\trans x = \nu^\trans X x$.

\paragraph{Astuce du noyau}
Dans le cas non, linéaire il existe une astuce du noyau. Il est bon de le savoir mais je ne vais pas en parler ici.

\sect{De la régression à la classification}

\subs{Régression logistique}

Jusqu'à maintenant, nous avons considéré $y \in \R$. On se place désormais dans le cas où $y \in \{0, 1\}$. Par exemple $y = 1$ lorsque le patient a une maladie et $y = 0$ sinon. Dans ce cas c'est généralement une  mauvaise idée d'utiliser un régression linéaire.

\begin{center}
	\begin{tikzpicture}[scale=2, thick, >={latex}]
		\draw[->] (-0.2, 0) -- (2.3, 0);
		\draw[->] (0, -0.2) -- (0, 1.3);
		\draw[fill=blue] (0.2, 0) circle(0.04);
		\draw[fill=blue] (0.3, 0) circle(0.04);
		\draw[fill=blue] (0.4, 0) circle(0.04);
		\draw[fill=blue] (0.5, 0) circle(0.04);
		\draw[fill=blue] (0.6, 1) circle(0.04);
		\draw[fill=blue] (0.7, 1) circle(0.04);
		\draw[fill=blue] (0.8, 1) circle(0.04);
		\draw[fill=blue] (0.9, 1) circle(0.04);
		\draw[fill=blue] (2, 1) circle(0.04);
		\draw[domain=0:1.8, smooth, variable=\x, red] plot ({\x}, {0.107+0.631*\x});
	\end{tikzpicture}
\end{center}

On utilise alors la régression logistique.

\DEF{
	La fonction \textbf{sigmoïde (logistique)} est définie par :
	$$ sig(z) = \dfrac{1}{1 + e^{-z}} $$
	\vspace{-3mm}
	Et sa fonction réciproque, la fonction \textbf{logit} est définie par :
	$$ logit(p) = \ln \left( \dfrac{p}{1-p} \right) $$
	\vspace{-6mm}
}

Cette fonction nous permet de définir le nouveau modèle :
\begin{center}
	\boldmath \fbox{$ \displaystyle h_\theta(x) = g \left( \theta^\trans x \right) = \dfrac{1}{1 + e^{-\theta^\trans x}} $}
\end{center}

\begin{center}
	\begin{tikzpicture}[scale=2, thick, >={latex}]
	\draw[->] (-0.2, 0) -- (2.3, 0);
	\draw[->] (0, -0.2) -- (0, 1.3);
	\draw[fill=blue] (0.2, 0) circle(0.04);
	\draw[fill=blue] (0.3, 0) circle(0.04);
	\draw[fill=blue] (0.4, 0) circle(0.04);
	\draw[fill=blue] (0.5, 0) circle(0.04);
	\draw[fill=blue] (0.6, 1) circle(0.04);
	\draw[fill=blue] (0.7, 1) circle(0.04);
	\draw[fill=blue] (0.8, 1) circle(0.04);
	\draw[fill=blue] (0.9, 1) circle(0.04);
	\draw[fill=blue] (2, 1) circle(0.04);
	\draw[domain=0.3:2, smooth, variable=\x, greenTikz, ultra thick] plot ({\x}, {1 / (1 + exp(-30.1*\x+16.55))});
	\draw[ultra thick, greenTikz] (0, 0) -- (0.3, 0);
	\end{tikzpicture}
\end{center}

Ainsi si $\theta^\trans > 0$ alors $h_\theta(x) > \frac{1}{2}$ et on pose $\hat{y} = 1$ sinon on pose $\hat{y} = 0$. On peut voir $h_\theta(x)$ comme une probabilité tel que $\Pp(y = 1|x, \theta) = h_\theta(x)$. \\
La méthode des moindres carrées est adaptée pour la régression mais pour la classification on préfère maximiser la vraisemblance puisque $h_\theta$ représente une probabilité.
$$ \Pp(y = 1 | x, \theta) = h_\theta(x) \qquad \text{et} \qquad \Pp(y = 0 | x, \theta) = 1 - h_\theta(x) $$
Cela nous donne :
$$ \Pp(y | x, \theta) = h_\theta(x)^y \left( 1 - h_\theta(x) \right)^{1-y} $$
On en déduit ensuite la vraisemblance :
$$ L(\theta) = \prod_{i=1}^m \Pp(y^{(i)} | x^{(i)}, \theta) = \prod_{i=1}^m h_\theta(x^{(i)})^{y^{(i)}} \left( 1 - h_\theta(x^{(i)}) \right)^{1-y^{(i)}}$$
Comme dit précédemment, il est plus facile de maximiser la log-vraisemblance :
$$ l(\theta) = \sum_{i = 1}^m y^{(i)} \ln \left( h_\theta(x^{(i)}) \right) + (1 - y^{(i)}) \ln \left( 1 - h_\theta(x^{(i)}) \right) $$
De la même manière qu'en régression linéaire, on va faire une ascension de gradient (et non plus une descente car ici on cherche à maximiser une vraisemblance et non à minimiser une distance).
$$ \theta_j \gets \theta_j + \alpha \dfrac{\partial}{\partial \theta_j} l(\theta) $$
Il nous faut alors calculer le gradient :
$$ \dfrac{\partial}{\partial \theta_j} l(\theta) = \sum_{i = 1}^m \left( y^{(i)} - h_\theta(x^{(i)}) \right) x_j^{(i)} $$
On obtient finalement :
\begin{center}
	\boldmath \fbox{$ \displaystyle \theta_j \gets \theta_j + \alpha \sum_{i = 1}^m \left( y^{(i)} - h_\theta(x^{(i)}) \right) x_j^{(i)} $}
\end{center}
On obtient exactement la même solution que pour la méthode des moindres carrés.

\paragraph{C'est un modèle linéaire} Nous avons dit que $\hat{y}$ vaut 1 si et seulement si $h_\theta(x)$ est plus grand que $\frac{1}{2}$. Or $h_\theta(x) > \frac{1}{2} \Leftrightarrow \theta^\trans x > 0$. Derrière cette régression logistique il y a finalement un modèle linéaire où l'hyperplan $\theta^\trans x = 0$ sépare l'espace en deux.

\begin{center}
	\begin{tikzpicture}[thick, scale=0.8]
		\draw[fill, red] (0, 1) circle (0.1);
		\draw[fill, red] (1, 2) circle (0.1);
		\draw[fill, red] (2, 2) circle (0.1);
		\draw[fill, blue] (5, 4) circle (0.1);
		\draw[fill, blue] (3, 4) circle (0.1);
		\draw[fill, blue] (2, 5) circle (0.1);
		\draw[fill, blue] (2, 4) circle (0.1);
		\draw[greenTikz, ultra thick] (-0.5, 3.13) -- (5.5, 2.54)
			node[right, black] {$\theta^\trans x = 0$};
		\draw[gray!40] (-0.5, 0.5) rectangle (5.5, 5.5);
	\end{tikzpicture}
\end{center}

\subs{Méthode de Newton}

Il existe une méthode plus rapide que l'ascension de gradient. Cette méthode est la méthode de Newton qui consiste à trouver le zéro du gradient de la log-vraisemblance.

\begin{center}
	\begin{tikzpicture}[xscale=2.8, yscale=1.7, >={latex}]
		\draw[->] (-0.1, 0) -- (2.2, 0);
		\draw[->] (0, -0.3) -- (0, 2.1);
		\draw[domain=0.1:2, smooth, variable=\x, blue, thick]
			plot ({\x}, {-0.686 + \x * (4.72 + \x * (-7.3 + \x * (4.53 + \x * -0.858)))});
		\coordinate (A) at (1.800000, 0.000000);
		\coordinate (B) at (1.800000, 1.559280);
		\coordinate (C) at (1.164304, 0.000000);
		\coordinate (D) at (1.164304, 0.479580);
		\coordinate (E) at (0.498177, 0.000000);
		\coordinate (F) at (0.498177, 0.358360);
		\draw[red, thick] (A) -- node[right] {\footnotesize $f(\theta^0)$} (B)
			-- (C) -- node[below] {\footnotesize $\Delta$} (A);
		\draw[dotted, greenTikz, thick] (C) -- (D) -- (E) -- (F);
		\draw[greenTikz] (A) node {$\bullet$} node[below, black] {\small $\theta^0$};
		\draw[greenTikz] (B) node {$\bullet$};
		\draw[greenTikz] (C) node {$\bullet$} node[below, black] {\small $\theta^1$};
		\draw[greenTikz] (D) node {$\bullet$};
		\draw[greenTikz] (E) node {$\bullet$} node[below, black] {\small $\theta^2$};
		\draw[greenTikz] (F) node {$\bullet$};
		\draw[red] (0.2, 0) node {$\bullet$};
	\end{tikzpicture}
\end{center}

La relation de récurrence pour faire converger $\theta$ vers le zéro d'une fonction $f$ est alors :
$$ \theta^{t+1} = \theta^t - \Delta = \theta^t - \dfrac{f(\theta^t)}{f'(\theta^t)} $$
Dans notre cas la fonction $f$ est $l'$, le gradient de la log-vraisemblance. On obtient :
\begin{center}
	\boldmath \fbox{$ \displaystyle \theta^{t+1} = \theta^t - H^{-1} \nabla_\theta l(\theta^t) $}
\end{center}
Où $H$ est la matrice hessienne $ \displaystyle H = \begin{pmatrix}
	\dfrac{\partial^2 l}{\partial \theta_0^2} & \dots & \dfrac{\partial^2 l}{\partial \theta_0 \theta_n} \\
	\vdots & \ddots & \vdots \\
	\dfrac{\partial^2 l}{\partial \theta_n \theta_0} & \dots & \dfrac{\partial^2 l}{\partial \theta_n^2}
\end{pmatrix} $ et $\nabla_\theta l$ est le gradient de $l$.

\paragraph{Avantage} Pour un nombre raisonnable de features, et d'exemples d'entraînement, la méthode de Newton converge bien plus rapidement.
\paragraph{Désavantage} A chaque itération, on doit inverser la matrice hessienne de taille $(n+1) \times (n+1)$. Ainsi si $n$ est grand alors cette inversion est très coûteuse.
	\chapter{K plus proches voisins et apprentissage de métriques}

\myminitoc

\sect{Classification bayésienne}

Le classificateur bayésien prédit la classe optimale $y^*$ d'un exemple $x \in \X$ de la manière suivante :
$$ y^*(x) = \argmax_c \Pp(y_c | x) = \argmax_c \dfrac{\Pp(x | y_c) \Pp(y_c)}{\Pp(x)} = \argmax_c \Pp(x | y_c) \Pp(y_c) $$
Si le calcul de $y^*$ est possible alors le classificateur bayésien est optimal d'un point de vue probabiliste et l'erreur associée est l'erreur de Bayes $\epsilon_B$. \\
Malheureusement, les $\Pp(y_c)$ et les $\Pp(x | y_c)$ sont inconnus. Mais on peut les estimer avec notre ensemble d'entraînement $S$.

\paragraph{\boldmath $\Pp(y_c)$}
Un estimateur non biaisé pour les probabilité des classes est la fréquence d'observation dans l'ensemble $S$ :
$$ \hat{p}(y_c) = \dfrac{|S_c|}{|S|} $$
Où $S_c$ est le nombre d'exemples appartenant à la classe $y_c$ dans l'ensemble $S$.

\paragraph{\boldmath $\Pp(x | y_c)$}
On peut distinguer deux types d'approche.
\begin{itemize}
	\item Les \textbf{méthodes paramétriques} qui assument que $\Pp(x \ y_c)$ suit une certaine loi. Dans ce cas le problème consiste à estimer les paramètres de cette loi. C'est donc une maximisation de la vraisemblance.
	\item Les \textbf{méthodes non paramétriques} qui n'imposent aucune contraintes sur la distribution et pour laquelle les valeurs de $\Pp(x | y_c)$ sont estimés de manière local autour de chaque $x$. La seule supposition sera alors que la distribution est localement régulière.
\end{itemize}

Pour simplifier un peu, on va essayer d'estimer $\Pp(x)$ au lieu de la probabilité $\Pp(x | y_c)$ qui s'obtient simplement en conditionnant l'ensemble d'entraînement $S$ par la classe $y_c$. \\
On considère la probabilité $\mathcal{P}_V$ que $x$ soit dans un volume $V$.
$$ \mathcal{P}_V = \int_V p(x)dx $$
Comme on suppose que $p$ est localement régulier, $p$ varie peu dans le volume $V$ si le volume est suffisamment petit. Pour $x$ dans $V$, on a donc :
$$ \hat{\mathcal{P}}_V \simeq p(x) \times V $$
Mais on peut aussi évaluer $\mathcal{P}$ avec la proportion d'exemples d'entraînement dans $V$. On pose $k_V$ le nombre d'exemple d'entraînement dans $V$ et on a :
$$ \hat{\mathcal{P}}_V \simeq \dfrac{k_V}{m} $$
On en déduit alors, pour $V$ un voisinage de $x$, l'égalité suivante :
$$ \hat{p}(x) = \dfrac{k_V}{m V} $$

\PROP{
	Soit $x \in \X$, et $V_m$ un voisinage de $x$ dans un ensemble de $m$ exemples d'entraînement et $k_m$ la proportion d'exemples de l'ensemble qui sont dans le voisinage $V_m$. \\
	Alors $\hat{p}(x) = \dfrac{k_m}{m V_m}$ converge vers $p(x)$ lorsque $m$ tend vers l'infini si les trois conditions suivantes sont remplies :
	$$ \bullet \lim\limits_{m \rightarrow +\infty} V_m = 0 \qquad \qquad \bullet \lim\limits_{m \rightarrow +\infty} k_m = +\infty \qquad \qquad \bullet \lim\limits_{m \rightarrow +\infty} \dfrac{k_m}{m} = 0$$
	\vspace{-5mm}
}

\paragraph{K plus proches voisins}
Il en vient que les $k$ plus proches voisins satisfont ces propriétés. On fixe un nombre $k_m$ de voisins et on prend un volume $V_m$ qui contient $k_m$ voisins.

\begin{center}
	\begin{tikzpicture}[thick, scale=2.8]
		\draw[blue] (0.626252, 0.819381) node {$\bullet$};
		\draw[red] (0.500000, 0.500000) circle (0.343429);
		\draw[blue] (2.626252, 0.819381) node {$\bullet$};
		\draw[blue] (2.766462, 0.208311) node {$\bullet$};
		\draw[blue] (2.265342, 0.502120) node {$\bullet$};
		\draw[blue] (2.555533, 0.427031) node {$\bullet$};
		\draw[blue] (2.403969, 0.184887) node {$\bullet$};
		\draw[blue] (2.720170, 0.620170) node {$\bullet$};
		\draw[blue] (2.899990, 0.885093) node {$\bullet$};
		\draw[blue] (2.759190, 0.760827) node {$\bullet$};
		\draw[blue] (2.378418, 0.014533) node {$\bullet$};
		\draw[blue] (2.471300, 0.483678) node {$\bullet$};
		\draw[blue] (2.245568, 0.344532) node {$\bullet$};
		\draw[blue] (2.455625, 0.462233) node {$\bullet$};
		\draw[blue] (2.229553, 0.741057) node {$\bullet$};
		\draw[blue] (2.905409, 0.670901) node {$\bullet$};
		\draw[blue] (2.151966, 0.686839) node {$\bullet$};
		\draw[blue] (2.407978, 0.621513) node {$\bullet$};
		\draw[red] (2.500000, 0.500000) circle (0.152425);
		\draw[blue] (4.626252, 0.819381) node {$\bullet$};
		\draw[blue] (4.766462, 0.208311) node {$\bullet$};
		\draw[blue] (4.265342, 0.502120) node {$\bullet$};
		\draw[blue] (4.555533, 0.427031) node {$\bullet$};
		\draw[blue] (4.403969, 0.184887) node {$\bullet$};
		\draw[blue] (4.720170, 0.620170) node {$\bullet$};
		\draw[blue] (4.899990, 0.885093) node {$\bullet$};
		\draw[blue] (4.759190, 0.760827) node {$\bullet$};
		\draw[blue] (4.378418, 0.014533) node {$\bullet$};
		\draw[blue] (4.471300, 0.483678) node {$\bullet$};
		\draw[blue] (4.245568, 0.344532) node {$\bullet$};
		\draw[blue] (4.455625, 0.462233) node {$\bullet$};
		\draw[blue] (4.229553, 0.741057) node {$\bullet$};
		\draw[blue] (4.905409, 0.670901) node {$\bullet$};
		\draw[blue] (4.151966, 0.686839) node {$\bullet$};
		\draw[blue] (4.407978, 0.621513) node {$\bullet$};
		\draw[blue] (4.768263, 0.902579) node {$\bullet$};
		\draw[blue] (4.940295, 0.946740) node {$\bullet$};
		\draw[blue] (4.805056, 0.646762) node {$\bullet$};
		\draw[blue] (4.875020, 0.015122) node {$\bullet$};
		\draw[blue] (4.647337, 0.161052) node {$\bullet$};
		\draw[blue] (4.467361, 0.583071) node {$\bullet$};
		\draw[blue] (4.472650, 0.188425) node {$\bullet$};
		\draw[blue] (4.015142, 0.723071) node {$\bullet$};
		\draw[blue] (4.023239, 0.575116) node {$\bullet$};
		\draw[blue] (4.468487, 0.325476) node {$\bullet$};
		\draw[blue] (4.849614, 0.368821) node {$\bullet$};
		\draw[blue] (4.386648, 0.559078) node {$\bullet$};
		\draw[blue] (4.016434, 0.761438) node {$\bullet$};
		\draw[blue] (4.703460, 0.915233) node {$\bullet$};
		\draw[blue] (4.242006, 0.270871) node {$\bullet$};
		\draw[blue] (4.673976, 0.614177) node {$\bullet$};
		\draw[blue] (4.368877, 0.375802) node {$\bullet$};
		\draw[blue] (4.407748, 0.458017) node {$\bullet$};
		\draw[blue] (4.072252, 0.932249) node {$\bullet$};
		\draw[blue] (4.969016, 0.134656) node {$\bullet$};
		\draw[red] (4.500000, 0.500000) circle (0.127823);
		\draw[thin] (0, 0) rectangle (1, 1);
		\draw[thin] (2, 0) rectangle (3, 1);
		\draw[thin] (4, 0) rectangle (5, 1);
		\node at (2.5, -0.2) {$k_m = \sqrt{m}$};
	\end{tikzpicture}
\end{center}

\sect{K plus proches voisins}

Cet algorithme des K plus proches voisins est en faite une bonne approximation de la distribution $p(x)$.

\exe
Prenons la distribution de la loi normale centrée réduite $\mathcal{N}(0, 1)$ définie par :
$$ p(x) = \dfrac{1}{\sqrt{2 \pi}} e^{-\frac{1}{2} x^2} $$
Puis on prendra $k = \sqrt{m}$.
\begin{center}
	\begin{tikzpicture}[yscale=4.5, xscale=0.85, thick]
		\draw[domain=-3:3, smooth, red, variable=\x] plot ({\x}, {exp(-0.5*\x*\x) * 0.399});
		\draw[blue] (-3.000000, 0.030912) -- (-2.900000, 0.032274) -- (-2.800000, 0.033761) -- (-2.700000, 0.035391) -- (-2.600000, 0.037187) -- (-2.500000, 0.039176) -- (-2.400000, 0.041389) -- (-2.300000, 0.043866) -- (-2.200000, 0.046660) -- (-2.100000, 0.049833) -- (-2.000000, 0.053470) -- (-1.900000, 0.057678) -- (-1.800000, 0.062607) -- (-1.700000, 0.068323) -- (-1.600000, 0.075349) -- (-1.500000, 0.083985) -- (-1.400000, 0.094857) -- (-1.300000, 0.108964) -- (-1.200000, 0.127998) -- (-1.100000, 0.155090) -- (-1.000000, 0.196730) -- (-0.900000, 0.241254) -- (-0.800000, 0.181496) -- (-0.700000, 0.152069) -- (-0.600000, 0.167628) -- (-0.500000, 0.217352) -- (-0.400000, 0.309016) -- (-0.300000, 0.332237) -- (-0.200000, 0.451218) -- (-0.100000, 0.491503) -- (0.000000, 0.562929) -- (0.100000, 0.408311) -- (0.200000, 0.262201) -- (0.300000, 0.292019) -- (0.400000, 0.271548) -- (0.500000, 0.203632) -- (0.600000, 0.159348) -- (0.700000, 0.169086) -- (0.800000, 0.219810) -- (0.900000, 0.209003) -- (1.000000, 0.254695) -- (1.100000, 0.282040) -- (1.200000, 0.352625) -- (1.300000, 0.439682) -- (1.400000, 0.274791) -- (1.500000, 0.259323) -- (1.600000, 0.217551) -- (1.700000, 0.167746) -- (1.800000, 0.136498) -- (1.900000, 0.115063) -- (2.000000, 0.099447) -- (2.100000, 0.087563) -- (2.200000, 0.078216) -- (2.300000, 0.070672) -- (2.400000, 0.064455) -- (2.500000, 0.059244) -- (2.600000, 0.054812) -- (2.700000, 0.050997) -- (2.800000, 0.047679) -- (2.900000, 0.044766) -- (3.000000, 0.042188);
		\node at (0, -0.05) {$m=20$};
		
		\draw[domain=-3:3, smooth, red, variable=\x] plot ({7+\x}, {exp(-0.5*\x*\x) * 0.399});
		\draw[blue] (4.000000, 0.016478) -- (4.100000, 0.017946) -- (4.200000, 0.019702) -- (4.300000, 0.021839) -- (4.400000, 0.024495) -- (4.500000, 0.027887) -- (4.600000, 0.031643) -- (4.700000, 0.032597) -- (4.800000, 0.038892) -- (4.900000, 0.043361) -- (5.000000, 0.055259) -- (5.100000, 0.074456) -- (5.200000, 0.091214) -- (5.300000, 0.086586) -- (5.400000, 0.104425) -- (5.500000, 0.148818) -- (5.600000, 0.141035) -- (5.700000, 0.170552) -- (5.800000, 0.140915) -- (5.900000, 0.199459) -- (6.000000, 0.332479) -- (6.100000, 0.415482) -- (6.200000, 0.390153) -- (6.300000, 0.243308) -- (6.400000, 0.231031) -- (6.500000, 0.428888) -- (6.600000, 0.369657) -- (6.700000, 0.309421) -- (6.800000, 0.405798) -- (6.900000, 0.289492) -- (7.000000, 0.332063) -- (7.100000, 0.349575) -- (7.200000, 0.528778) -- (7.300000, 0.421831) -- (7.400000, 0.368973) -- (7.500000, 0.266194) -- (7.600000, 0.219743) -- (7.700000, 0.292974) -- (7.800000, 0.305682) -- (7.900000, 0.279585) -- (8.000000, 0.234051) -- (8.100000, 0.138772) -- (8.200000, 0.130446) -- (8.300000, 0.133778) -- (8.400000, 0.166943) -- (8.500000, 0.188931) -- (8.600000, 0.119329) -- (8.700000, 0.100467) -- (8.800000, 0.087022) -- (8.900000, 0.063306) -- (9.000000, 0.058376) -- (9.100000, 0.050152) -- (9.200000, 0.040153) -- (9.300000, 0.034502) -- (9.400000, 0.032542) -- (9.500000, 0.028280) -- (9.600000, 0.024798) -- (9.700000, 0.022079) -- (9.800000, 0.019897) -- (9.900000, 0.018108) -- (10.000000, 0.016614);
		\node at (7, -0.05) {$m=500$};
		
		\draw[domain=-3:3, smooth, red, variable=\x] plot ({14+\x}, {exp(-0.5*\x*\x) * 0.399});
		\draw[blue] (11.000000, 0.006779) -- (11.100000, 0.007873) -- (11.200000, 0.009383) -- (11.300000, 0.011256) -- (11.400000, 0.013063) -- (11.500000, 0.016750) -- (11.600000, 0.021479) -- (11.700000, 0.025188) -- (11.800000, 0.029294) -- (11.900000, 0.037798) -- (12.000000, 0.048504) -- (12.100000, 0.070106) -- (12.200000, 0.086664) -- (12.300000, 0.091432) -- (12.400000, 0.105619) -- (12.500000, 0.131789) -- (12.600000, 0.167379) -- (12.700000, 0.199576) -- (12.800000, 0.179463) -- (12.900000, 0.215490) -- (13.000000, 0.235025) -- (13.100000, 0.261005) -- (13.200000, 0.258727) -- (13.300000, 0.331118) -- (13.400000, 0.342982) -- (13.500000, 0.399206) -- (13.600000, 0.327205) -- (13.700000, 0.435013) -- (13.800000, 0.328604) -- (13.900000, 0.399489) -- (14.000000, 0.392978) -- (14.100000, 0.391679) -- (14.200000, 0.412969) -- (14.300000, 0.361395) -- (14.400000, 0.391842) -- (14.500000, 0.364244) -- (14.600000, 0.314217) -- (14.700000, 0.335372) -- (14.800000, 0.258383) -- (14.900000, 0.230056) -- (15.000000, 0.225464) -- (15.100000, 0.240329) -- (15.200000, 0.183084) -- (15.300000, 0.194312) -- (15.400000, 0.140430) -- (15.500000, 0.135825) -- (15.600000, 0.116266) -- (15.700000, 0.085234) -- (15.800000, 0.080030) -- (15.900000, 0.055827) -- (16.000000, 0.049774) -- (16.100000, 0.048143) -- (16.200000, 0.039489) -- (16.300000, 0.034667) -- (16.400000, 0.024548) -- (16.500000, 0.019633) -- (16.600000, 0.017155) -- (16.700000, 0.013106) -- (16.800000, 0.010784) -- (16.900000, 0.009069) -- (17.000000, 0.007646);
		\node at (14, -0.05) {$m=10000$};
	\end{tikzpicture}
\end{center}

\paragraph{k-NN}
On appelle le classificateur des $k$ plus proches voisins k-NN. On note $m_j$ le nombre d'exemples dans la classe $y_j$. De même on note $k_j$ le nombre d'exemples dans la classe $y_j$ qui sont dans l'hypersphère centrée en $x$ qui contient les $k$ plus proches voisins. On a donc :
$$ \sum_j m_j = m \qquad \text{et} \qquad \sum_j k_j = k $$
Ensuite on regarde ce que nous donne la classification bayésienne :
$$ h(x) = \argmax_j \dfrac{\hat{p}(x | y_j) \hat{p}(y_j)}{\hat{p}(x)} = \argmax_j \dfrac{\dfrac{k_j}{m_j \times V} \times \dfrac{m_j}{m}}{\dfrac{k}{m \times V}} = \argmax_j \dfrac{k_j}{k} $$

\begin{center}
	\begin{tikzpicture}[scale=0.8, >={latex}]
		\draw[->] (-0.4, -0.2) -- (7, -0.2);
		\draw[->] (-0.4, -0.2) -- (-0.4, 5);
		\node[yellow] at (0, 4) {\Large $\bullet$};
		\node[yellow] at (1, 2.3) {\Large $\bullet$};
		\node[yellow] at (1.5, 4.2) {\Large $\bullet$};
		\node[yellow] at (2, 3) {\Large $\bullet$};
		\node[yellow] at (2.3, 4.3) {\Large $\bullet$};
		\node[red] at (2.5, 2.6) {\Large $\star$};
		\node[purple] at (2.8, 2) {\Large $\bullet$};
		\node[purple] at (3.3, 3) {\Large $\bullet$};
		\node[purple] at (4.8, 1.6) {\Large $\bullet$};
		\node[purple] at (6, 2.5) {\Large $\bullet$};
		\node[purple] at (6, 0.5) {\Large $\bullet$};
		
		\draw[dashed] (2.5, 2.6) circle (1.1);
		\draw[dashed] (2.5, 2.6) circle (2.05);
		\node[below] at (2.5, 1.5) {\footnotesize $k=3$};
		\node[below] at (2.5, 0.6) {\footnotesize $k=6$};
		\node[yellow!100] at (5.5, 4.5) {Classe A};
		\node[purple!100] at (5.5, 4) {Classe B};
	\end{tikzpicture}
\end{center}

\PROP{
	L'erreur de généralisation $\epsilon_{1NN}$ du 1-plus proche voisin est bornée par deux fois l'erreur de Bayes $\epsilon_B$ lorsque $m$ tend vers l'infini :
	$$2 \epsilon_B - 2 \epsilon_B^2 \leqslant \epsilon_{1NN} \leqslant 2 \epsilon_B - \epsilon_B^2$$
	\vspace{-5mm}
}

\dem
Soit $x \in \X$ et $y$ sa classe c'est à dire $y = \argmax_{y_j} p(y_j | x)$. L'erreur de Bayes est la suivante :
$$ \epsilon_B(x) = \sum_{y_j \neq y} p(y_j | x) $$
L'erreur de l'algorithme 1-plus proche voisin est la suivante :
$$ \epsilon_{1NN} = 1 - \sum_j p(y_j | x) p(y_j | x') $$
où $x'$ est le plus proche voisin de $x$. Quand $m$ tend vers l'infini, $x'$ tend vers $x$. Cela nous donne :
$$ \epsilon_{1NN} \simeq 1 - \sum_j p(y_j | x)^2 = 1 - p(y|x)^2 - \sum_{y_j \neq y} p(y_j | x)^2 $$
Or $p(y | x) = 1 - \epsilon_B$ et on peut borner la somme qui intervient dans $\epsilon_{1NN}$. En effet soit un ensemble $\{ v_i \}$ de valeurs positives tel que $\sum_i v_i = a$. On a :
$$ \left( \sum_i v_i \right)^2 = a^2 $$
On développe la somme au carré et on obtient :
$$ 0 \leqslant \sum_i v_i^2 = a^2 - 2 \sum_{i \neq j} v_i v_j \leqslant a^2 $$.
Dans notre cas, cela donne :
$$ 0 \leqslant \sum_{y_j \neq y} p(y_j | x)^2 \leqslant \epsilon_B^2 $$
Finalement en reprenant notre expression de $\epsilon_{1NN}$ on obtient :
$$ 1 - (1 - \epsilon_B)^2 - \epsilon_B^2 \leqslant \epsilon_{1NN} \leqslant 1 - (1 - \epsilon_B)^2 $$
Ce qui nous donne les inégalités souhaitées.
\findem

\paragraph{Effet de $k$} De plus l'erreur asymptotique diminue lorsque $k$ augmente.
$$ \epsilon_B \leqslant \epsilon_{kNN} \leqslant \epsilon_{(k-1)NN} \leqslant ... \leqslant \epsilon_{1NN} \leqslant 2 \epsilon_B $$
Le graphique ci dessous nous montre l'erreur pour différentes valeurs de $k$ pour classification binaire ($|\Y| = 2$).
\begin{center}
	\begin{tikzpicture}[yscale=7, xscale=9, thick, >={latex}]
		\draw[->] (0, 0) -- (0.55, 0);
		\draw[->] (0, 0) -- (0, 0.55);
		\draw[red, ultra thick] (0, 0) -- (0.5, 0.5);
		\draw[red] (0.32, 0.1) node {\footnotesize Erreur de Bayes};
		\draw[->, red] (0.32, 0.14) -- (0.27, 0.26);
		\draw[] (0.25, 0.02) -- (0.25, -0.02) node[below] { \footnotesize 0.25};
		\draw[] (0.5, 0.025) -- (0.5, -0.025) node[below] { \footnotesize 0.5};
		\draw[] (0.019, 0.5) -- (-0.019, 0.5) node[left] { \footnotesize 0.5};
		\draw[] (0.015, 0.25) -- (-0.015, 0.25) node[left] { \footnotesize 0.25};
		\draw[blue] (0.000000, 0.000000) -- (0.016667, 0.032778) -- (0.033333, 0.064444) -- (0.050000, 0.095000) -- (0.066667, 0.124444) -- (0.083333, 0.152778) -- (0.100000, 0.180000) -- (0.116667, 0.206111) -- (0.133333, 0.231111) -- (0.150000, 0.255000) -- (0.166667, 0.277778) -- (0.183333, 0.299444) -- (0.200000, 0.320000) -- (0.216667, 0.339444) -- (0.233333, 0.357778) -- (0.250000, 0.375000) -- (0.266667, 0.391111) -- (0.283333, 0.406111) -- (0.300000, 0.420000) -- (0.316667, 0.432778) -- (0.333333, 0.444444) -- (0.350000, 0.455000) -- (0.366667, 0.464444) -- (0.383333, 0.472778) -- (0.400000, 0.480000) -- (0.416667, 0.486111) -- (0.433333, 0.491111) -- (0.450000, 0.495000) -- (0.466667, 0.497778) -- (0.483333, 0.499444) -- (0.500000, 0.500000);
		\draw[blue] (0.52, 0.4) -- (0.56, 0.4) node[right] {\small $k=1$};
		\draw[purple] (0.000000, 0.000000) -- (0.016667, 0.017463) -- (0.033333, 0.036375) -- (0.050000, 0.056525) -- (0.066667, 0.077709) -- (0.083333, 0.099730) -- (0.100000, 0.122400) -- (0.116667, 0.145537) -- (0.133333, 0.168968) -- (0.150000, 0.192525) -- (0.166667, 0.216049) -- (0.183333, 0.239389) -- (0.200000, 0.262400) -- (0.216667, 0.284945) -- (0.233333, 0.306894) -- (0.250000, 0.328125) -- (0.266667, 0.348523) -- (0.283333, 0.367982) -- (0.300000, 0.386400) -- (0.316667, 0.403685) -- (0.333333, 0.419753) -- (0.350000, 0.434525) -- (0.366667, 0.447931) -- (0.383333, 0.459908) -- (0.400000, 0.470400) -- (0.416667, 0.479360) -- (0.433333, 0.486746) -- (0.450000, 0.492525) -- (0.466667, 0.496672) -- (0.483333, 0.499167) -- (0.500000, 0.500000);
		\draw[purple] (0.52, 0.35) -- (0.56, 0.35) node[right] {\small $k=3$};
		\draw[] (0.000000, 0.000000) -- (0.016667, 0.016667) -- (0.033333, 0.033338) -- (0.050000, 0.050030) -- (0.066667, 0.066781) -- (0.083333, 0.083650) -- (0.100000, 0.100713) -- (0.116667, 0.118056) -- (0.133333, 0.135771) -- (0.150000, 0.153940) -- (0.166667, 0.172633) -- (0.183333, 0.191897) -- (0.200000, 0.211749) -- (0.216667, 0.232171) -- (0.233333, 0.253107) -- (0.250000, 0.274464) -- (0.266667, 0.296105) -- (0.283333, 0.317860) -- (0.300000, 0.339523) -- (0.316667, 0.360861) -- (0.333333, 0.381615) -- (0.350000, 0.401516) -- (0.366667, 0.420283) -- (0.383333, 0.437639) -- (0.400000, 0.453314) -- (0.416667, 0.467055) -- (0.433333, 0.478635) -- (0.450000, 0.487858) -- (0.466667, 0.494564) -- (0.483333, 0.498635) -- (0.500000, 0.500000);
		\draw[] (0.52, 0.3) -- (0.56, 0.3) node[right] {\small $k=9$};
		\draw[greenTikz] (0.000000, 0.000000) -- (0.016667, 0.016667) -- (0.033333, 0.033333) -- (0.050000, 0.050000) -- (0.066667, 0.066667) -- (0.083333, 0.083333) -- (0.100000, 0.100000) -- (0.116667, 0.116667) -- (0.133333, 0.133335) -- (0.150000, 0.150006) -- (0.166667, 0.166686) -- (0.183333, 0.183388) -- (0.200000, 0.200137) -- (0.216667, 0.216977) -- (0.233333, 0.233975) -- (0.250000, 0.251225) -- (0.266667, 0.268844) -- (0.283333, 0.286963) -- (0.300000, 0.305703) -- (0.316667, 0.325148) -- (0.333333, 0.345309) -- (0.350000, 0.366087) -- (0.366667, 0.387241) -- (0.383333, 0.408374) -- (0.400000, 0.428930) -- (0.416667, 0.448226) -- (0.433333, 0.465495) -- (0.450000, 0.479952) -- (0.466667, 0.490877) -- (0.483333, 0.497686) -- (0.500000, 0.500000);
		\draw[greenTikz] (0.52, 0.25) -- (0.56, 0.25) node[right] {\small $k=27$};
		\node[below left] at (0, 0) {0};
	\end{tikzpicture}
\end{center}
Comme la propriété précédente n'est valide que lorsque $m$ est suffisamment grand. Il nous faut alors un compromis pour que l'inégalité tienne sans que $m$ soit trop grand. On prend généralement $k = \sqrt{\dfrac{m}{|\Y|}}$.

\paragraph{Problèmes}
Pour faire face à la malédiction de la dimensionnalité, il faut réduire la dimension avec des algorithmes comme PCA. Pour converger l'algorithme des k-plus proches voisins a besoin de beaucoup d'exemples. Cependant, un large nombre d'exemples d'entraînement implique une complexité spatiale et temporelle élevée. Pour résoudre ça, on peut utiliser les deux stratégies suivantes :
\begin{itemize}
	\item Réduire la taille $S$ en ne gardant que les exemples les plus révélateurs.
	\item Simplifier le calcul des plus proches voisins.
\end{itemize}

\paragraph{Étape préliminaire}
La première étape consiste en la suppression des exemples qui sont aberrants ou qui sont dans la région de l'erreur de Bayes.
\begin{center}
	\begin{algorithm}[H]
		\KwIn{$S$}
		\KwOut{$S_{cleaned}$}
		Séparer aléatoirement $S$ en deux parties $S_1$ et $S_2$\;
		\Repeat{stabilisation de $S_1$ et $S_2$}{
			Classer $S_1$ en utilisant 1-NN avec $S_2$\;
			Supprimer de $S_1$ les instances mal classées\;
			Classer $S_2$ en utilisant 1-NN avec le nouvel ensemble $S_1$\;
			Supprimer de $S_2$ les instances mal classées\;
		}
		\KwResult{$S_{cleaned} = S_1 \cup S_2$}
		\caption{Réduction de données}
	\end{algorithm}
	
	\includegraphics[scale=0.35]{big_samples.png}
	\includegraphics[scale=0.35]{cleaned_samples.png}
\end{center}

\paragraph{Seconde étape}
Puis on supprime les exemples qui ne sont pas très révélateurs.
\begin{center}
	\begin{algorithm}[H]
		\KwIn{$S$}
		\KwOut{STORAGE}
		STORAGE $\gets \emptyset$\; 
		Choisir aléatoirement un exemple de $S$ et le mettre dans STORAGE\;
		\Repeat{stabilisation de STORAGE}{
			\For{$x \in S$} {
				\If{$x$ n'est pas correctement classé en utilisant 1-NN avec STORAGE}{
					STORAGE $\gets$ STORAGE $\cup \; \{x\}$
				}
			}
		}
		\KwResult{STORAGE}
		\caption{Plus proche voisin condensé (CNN)}
	\end{algorithm}
	
	\includegraphics[scale=0.35]{cleaned_samples.png}
	\includegraphics[scale=0.35]{condensed_samples.png}
\end{center}

\paragraph{Augmenter la rapidité}
En 2D ou en 3D on peut utiliser des diagrammes de Voronoi ou encore des graphes de proximité.
En plus grande dimension on peut utiliser les structures de données suivantes : ball-trees, kd-trees, metric-trees, quadtrees et R-trees.

\paragraph{Classification naïve bayésienne}
La classification naïve bayésienne suppose en plus que les features sont indépendantes.
$$ \hat{y}(x_1, ..., x_n) = \argmax_c \prod_{i=1}^n p(x_i | y_c).p(y_c) $$
Cette classification est beaucoup moins sensible à la dimension comme chaque probabilité peut être estimée comme une distribution de dimension 1. Il n'y a alors plus de malédiction de la dimensionnalité. De plus cela rend plus lisse la probabilité obtenue.

\paragraph{Question} Comment peut-on améliorer la qualité de la métrique utilisée dans k-NN ?

\sect{Apprentissage de métriques}

On commence par rappeler ce q'est une distance sur un espace $X$.

\DEF{
	Une \textbf{métrique/distance} sur un espace $X$ est une fonction positive $d : X \times X \rightarrow \R_+$ qui vérifie les conditions suivantes :
	\begin{itemize}
		\item Positivité : $d(x, y) \geqslant 0$
		\item Séparation : $d(x, y) = 0 \Leftrightarrow x = y$
		\item Symétrie : $d(x, y) = d(y, x)$
		\item Inégalité triangulaire : $d(x, z) \leqslant d(x, y) + d(y, z)$
	\end{itemize}
}

Les normes $l_p$ sont des distances. La norme $l_1$ est appelée \textbf{distance de Manhattan}, la norme $l_2$ est appelée \textbf{distance euclidienne} et la norme $l_\infty$ est appelée \textbf{distance de Tchebychev}. \\
Mais comment choisir la bonne métrique ? L'idée est d'apprendre une métrique qui assigne une faible distance aux paires d'exemples qui sont sémantiquement similaire (par exemple qui ont la même classe dans le cas de la classification supervisée) et qui assigne une distance élevée aux paires d'exemples qui ne sont sémantiquement différents.

\begin{center}
	\begin{tikzpicture}[scale=4, thick, >={latex}]
		\draw[fill=red] (0.45, 0.2) circle (0.06);
		\draw[fill=red] (0.13, 0.6) circle (0.06);
		\draw[fill=blue] (0.22, 0.3) circle (0.06);
		\draw[fill=blue] (0.52, 0.52) circle (0.06);
		\draw[fill=blue] (0.2, 0.85) circle (0.06);
		\draw[fill=blue] (0.87, 0.75) circle (0.06);
		\draw (0, 0) rectangle (1, 1);
		\draw[fill=red] (2.08, 0.55) circle (0.06);
		\draw[fill=red] (2.16, 0.4) circle (0.06);
		\draw[fill=blue] (2.45, 0.45) circle (0.06);
		\draw[fill=blue] (2.6, 0.52) circle (0.06);
		\draw[fill=blue] (2.52, 0.65) circle (0.06);
		\draw[fill=blue] (2.77, 0.61) circle (0.06);
		\draw[dashed] (2.595, 0.555) circle (0.255);
		\draw[dashed] (2.595, 0.555) circle (0.39);
		\draw (2, 0) rectangle (3, 1);
		
		\node at (1.5, 0.55) {\large Apprentissage};
		\draw[line width=6, ->] (1.05, 0.42) -- (1.95, 0.42);
	\end{tikzpicture}
\end{center}

\DEF{
	La \textbf{distance de Mahalanobis} est définie de la manière suivante :
	$$ d_M(x, x') = \sqrt{(x - x')^\trans M (x - x')} $$
	Où $M \in \R^{d \times d}$ est une matrice symétrique définie positive.
}

Cette distance vient du cas où $x$ et $x'$ suivent la même distribution de matrice de covariance $\Sigma$ et dans ce cas $M = \Sigma^{-1}$. \\
On peut décider de prendre $M$ uniquement positive et pas définie positive, c'est à dire autoriser $M$ à avoir des valeurs propres nulles. Cela permet de faire une réduction de dimension. \\
De plus comme $M$ est symétrique positive on peut l'écrire de la manière suivante : $M = L^\trans L$. Dans ce cas en posant $\tilde{x} = Lx$, on a :
$$ \begin{array}{lll}
d_M(x, x')
& = & \sqrt{(x - x')^\trans L^\trans L (x - x')} \\
& = & \sqrt{(Lx - Lx')^\trans (Lx - Lx')} \\
& = & \sqrt{(\tilde{x} - \tilde{x}')^\trans (\tilde{x} - \tilde{x}')} \\
& = & d_{euc}(\tilde{x}, \tilde{x}')
\end{array} $$

\paragraph{Apprentissage}
Désormais, étant donnée une métrique paramétrisée par $M$, on va essayer de résoudre le problème d'optimisation suivant:
$$ M^*=\argmin_{M \succeq 0} l(M, \mathcal{S}, \mathcal{D}, \mathcal{R}) + \lambda R(M) $$
Où $l$ est une fonction de perte qui pénalise des contraintes violées, $R$ est un régularisateur de $M$ et $\lambda$ est un paramètre qui permet de contrôler l'importance de la régularisation. L'état de l'art se différencie alors par les choix de la fonction de perte, des contraintes et du régularisateur. \\
Les contraintes sont les suivantes :
\begin{itemize}
	\item $\displaystyle \mathcal{S} = \left\{ (x, x')~|~x \text{ et } x' \text{ sont similaires} \right\}$
	\item $\displaystyle \mathcal{D} = \left\{ (x, x')~|~x \text{ et } x' \text{ sont éloignés} \right\}$
	\item $\displaystyle \mathcal{R} = \left\{ (x, x', x'')~|~x \text{ est plus proche de } x' \text{ que de } x'' \right\}$
\end{itemize}
Les régularisateurs usuels sont les suivants :
\begin{itemize}
	\item La norme de Frobenius (simple et classique) $\| M \|_\mathcal{F}^2 = \sum_{i, j} M_{ij}^2$.
	\item La sélections de features avec la norme mixe $L_{2, 1}$ : $\| M \|_{2, 1} = \sum_i \| M_i \|_2$. \\ Cette norme est bien convexe mais elle n'est pas lisse.
	\item La favorisation des matrices de rang faible pour la réduction de dimension \\
		$\| M \|_* = tr(M) = \sum_i \sigma_i(M) $.
	\item L'utilisation de la LogDet divergence.
\end{itemize}

\paragraph{LMNN}
L'algorithme LMNN considère les ensembles de contraintes suivants :
$$ \mathcal{S} = \left\{ (x_i, x_j)~|~y_i = y_j \text{ et } x_j \text{ appartient aux } k \text{ plus proches voisins de } x_i \right\} $$
$$ \mathcal{R} = \left\{ (x_i, x_j, x_k)~|~(x_i, x_j) \in \mathcal{S}, \, y_i \neq y_k \right\} $$
La formulation stricte est alors la suivante :
$$ \min_{M \succeq 0} \quad \sum_{(x_i, x_j) \in \mathcal{S}} d_M^2(x_i, x_j) $$
\vspace{-4mm}
$$ s.t. \quad \forall (x_i, x_j, x_k) \in \mathcal{R}, \; d_M^2(x_i, x_k) - d_M^2(x_i, x_j) \geqslant 1 $$
La formulation douce est la suivante :
$$ \min_{M \succeq 0} \quad (1 - \mu) \sum_{(x_i, x_j) \in \mathcal{S}} d_M^2(x_i, x_j) + \mu \sum_{i, j, k} \xi_{ijk} $$
\vspace{-4mm}
$$ s.t. \quad \forall (x_i, x_j, x_k) \in \mathcal{R}, \; d_M^2(x_i, x_k) - d_M^2(x_i, x_j) \geqslant 1 - \xi_{ijk} $$
Où $\mu$ contrôle la contrepartie entre attirer les exemples proches et éloigner les exemples d'étiquettes différentes. \\
L'avantage de cet apprentissage est qu'il est convexe et qu'il possède un solveur efficace même pour un large nombre de contraintes. Le désavantage est qu'il est soumis à l'overfitting lorsque la dimension est élevée.

\paragraph{ITML} L'Information-Thoretical Metric Learning introduit une régularisation utilisant la divergence LogDet définie par :
$$ D_{id}(M, M_0) = tr \left (MM_0^{-1} \right) - \log \det \left( MM_0^{-1} \right) - d $$
Où $d$ est la dimension et $M_0$ est une matrice symétrique positive dont on veut rester proche. L'ITML est formulé de la manière suivante :
$$ \min_{M \succeq 0} \quad D_{id}(M, M_0) $$
\vspace{-4mm}
$$ s.t. \quad \left\{ \begin{array}{ll}
	\forall (x_i, x_j) \in \mathcal{S} & d_M^2(x_i, x_j) \leqslant u \\
	\forall (x_i, x_j) \in \mathcal{D} & d_M^2(x_i, x_j) \geqslant v
\end{array} \right.$$

\paragraph{Non-linéarité}
LMNN et ITML génèrent des transformations linéaires qui sont incapable ce capturer des structures non linéaires dans les données. Comme solution à ce problème on peut utiliser des noyaux comme KPCA (PCA dans un espace noyau) ou des métriques locales. Pour les métriques locales on peut par exemples partitionner notre ensemble en C clusters puis apprendre une distance de Mahalanobis sur chacun de ces clusters.
	
\end{document}